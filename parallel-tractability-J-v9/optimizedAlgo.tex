\section{Further Optimizing DHL$(\circ)$ Materialization}
\label{sec:practicalAlg}

In this section, we first discuss why Algorithm~$\mathsf{A}_{\text{opt}}$ can hardly work in practice.
In order to make Algorithm~$\mathsf{A}_{\text{opt}}$ more practical,
we propose to modify Algorithm~$\mathsf{A}_{\text{opt}}$ and give an algorithm variant Algorithm~$\mathsf{A}_{\text{prc}}$.
We show that Algorithm~$\mathsf{A}_{\text{prc}}$ can also be restricted to an \texttt{NC} version
when materializing DHL$(\circ)$ ontologies that satisfy both the simple-concept
and the simple-role restriction.

\subsection{Reducing Computing Space}

In previous sections, Algorithm~$\mathsf{A}_{\text{opt}}$ is mainly used for theoretical analysis.
However, this algorithm can hardly work in practice due to its inherently
high requirement of computing space. Specifically,
Algorithm~$\mathsf{A}_{\text{opt}}$ constructs a materialization graph of the given
datalog program $P$ by checking all possible rule
instantiations in $P^*$.
One can check that $|P^*|$ could be the quadratic or cubic in the number of constants occurring in $P$.
Recall the analysis of the parallel complexity of Algorithm~$\mathsf{A}_{\text{bsc}}$ and Algorithm~$\mathsf{A}_{\text{opt}}$.
The number of constants is at most $|\textbf{I}|e$ where $e$ denotes the maximum arity of any predicate in $\textbf{I}$.
Consider a datalog rule of the form `$R(x,y),A(y)\rightarrow B(x)$'.
Since there are two variables in this rule, the number of all possible rule instantiations
is $(|\textbf{I}|e)^2$.
Similarly, for a datalog rule rewritten from an axiom of the form (R3) or (R4),
the number of all possible rule instantiations is $(|\textbf{I}|e)^3$.
Undoubtedly, a plain implementation of Algorithm~$\mathsf{A}_{\text{opt}}$ would be slow
when the target datalog program or ontology
tends to be large in size.
On the other hand, from Examples~\ref{exp:mg}, \ref{exp:dllite} and \ref{exp:dhl},
we can observe that the rule instantiations used for the construction
of materialization graphs always cover a small part of $P^*$ with respect to the target datalog program $P$.
Thus, we consider reducing the computing space of Algorithm~$\mathsf{A}_{\text{opt}}$
by narrowing down the scope of rule instantiations to be checked.
Our strategy is to restrict that, \emph{in each iteration of Algorithm~$\mathsf{A}_{\text{opt}}$,
each of the checked rule instantiations should involve at least one body atom that has been
added to the constructed materialization graph}.
Since DHL$(\circ)$
is our focus, we explain how to apply the above strategy in DHL$(\circ)$ materialization
as follows.

We first consider a datalog rule of the
form `$A(x)\rightarrow B(x)$' that corresponds to (T1). The above strategy requires that,
in each iteration, Algorithm~$\mathsf{A}_{\text{opt}}$ can only check the rule instantiations
of the form `$A(a)\rightarrow B(a)$' where $A(a)$ has been added to the constructed materialization
graph $\mathcal{G}$. If there are $n$ atoms of the form $A(x)$ that have been added to $\mathcal{G}$,
the number of all checked rule instantiations of the above rule is also $n$, instead of $(|\textbf{I}|e)$.
This is because, for each assertion $A(a)$,
the datalog rule `$A(x)\rightarrow B(x)$'
has only one rule instantiation `$A(a)\rightarrow B(a)$' where
the variable $x$ is substituted by the constant $a$.
The cases of (R1) and (R2) can be analyzed similarly.
We now consider datalog rules that have two body atoms, i.e.,
datalog rules of the form `$R(x,y),A(y)\rightarrow B(x)$' (see (T3)),
`$A_1(x),A_2(x)\rightarrow B(x)$' (see (T2)) and `$R_1(x,y),R_2(y,z)\rightarrow R(x,z)$' (see (R3) and (R4)).
We require that, for each rule instantiation of these rules checked by Algorithm~$\mathsf{A}_{\text{opt}}$,
at least one body atom has been added to the constructed materialization
graph $\mathcal{G}$; thus, it can be checked that, in each iteration of Algorithm~$\mathsf{A}_{\text{opt}}$,
the number of checked rule instantiations
is at most $k(|\textbf{I}|e)$ where $k$ is the number of atoms that have been
added to $\mathcal{G}$.
Note that, for rule instantiations of two body atoms,
we do not require that both of these two body atoms have been added to
the constructed materialization graph.
Otherwise, the algorithm would
perform as Algorithm~$\mathsf{A}_{\text{bsc}}$ and, thus, the
optimizations used in Algorithm~$\mathsf{A}_{\text{opt}}$ cannot further work.



\subsection{Further Optimizing Algorithm~$\mathsf{A}_{\text{opt}}$}


We use the above method of narrowing down the scope of rule instantiations to
modify Algorithm~$\mathsf{A}_{\text{opt}}$ and obtain an algorithm variant Algorithm~$\mathsf{A}_{\text{prc}}$.
Recall that,
in each iteration of \ref{alg3:updateG}, Algorithm~$\mathsf{A}_{\text{opt}}$
checks all possible rule instantiations and computes an \texttt{rch} relation
and its transitive closure to determine the existence of SWD paths.
We let Algorithm~$\mathsf{A}_{\text{prc}}$ conduct the same work with
narrowing down the scope of rule instantiations to be checked as well. This can be described
by the following algorithm.\\

\noindent\texttt{Algorithm~$\mathsf{PRC}$}. This algorithm has as inputs (1)
a DHL$(\circ)$ ontology $\mathcal{O}=\langle\mathcal{T},\mathcal{R},\mathcal{A}\rangle$
and its datalog program $P=\langle R, \textbf{I}\rangle$;
(2) a (partial) materialization graph $\mathcal{G}=\langle V,E\rangle$ that is constructed from $P$.
This algorithm outputs an \texttt{rch} relation $S_{\!\textit{\tiny rch}}$ that
is computed as follows:

\begin{enumerate}[leftmargin=4ex,label=$\bullet$]
\item add \texttt{rch}$(A(a),B(a))$ to $S_{\textit{\!\tiny rch}}$ where $A(a)\rightarrow B(a)\in P^*$, $\mathcal{O}\models A\sqsubseteq_* B$ and $A(a)\in V$;

\item add \texttt{rch}$(A(b),B(a))$ to $S_{\textit{\!\tiny rch}}$ where $R(a,b),A(b)\rightarrow B(a)\in P^*$, $\exists R.A\sqsubseteq B\in\mathcal{T}$ and $R(a,b)\in V$;

\item add \texttt{rch}$(A_2(a),B(a))$ to $S_{\textit{\!\tiny rch}}$ where $A_1(a),A_2(a)\rightarrow B(a)\in P^*$,
    $A_1\sqcap A_2\sqsubseteq B\in\mathcal{T}$ and $A_1(a)\in V$;

\item add \texttt{rch}$(A_1(a),B(a))$ to $S_{\textit{\!\tiny rch}}$ where $A_1(a),A_2(a)\rightarrow B(a)\in P^*$,
    $A_1\sqcap A_2\sqsubseteq B\in\mathcal{T}$ and $A_2(a)\in V$;

\item add \texttt{rch}$(R(a,b)$, $S(a,b))$ to $S_{\textit{\!\tiny rch}}$ where $R(a,b)\rightarrow S(a,b)\in P^*$,
    $\mathcal{O}\models R\sqsubseteq_* S$ and $R(a,b)\in V$;

\item add \texttt{rch}$(R(a,b)$, $S(b,a))$ to $S_{\textit{\!\tiny rch}}$ where $R(a,b)\rightarrow S(b,a)\in P^*$,
    $\mathcal{O}\models R\sqsubseteq_* S^-$ and $R(a,b)\in V$;

\item add \texttt{rch}$(R_2(b,c),R_3(a,c))$ to $S_{\textit{\!\tiny rch}}$ where $R_1(a,b),R_2(b,c)\rightarrow R_3(a,c)\in P^*$,
    $R_1\circ R_2\sqsubseteq R_3\in\mathcal{R}$ (the case
    where $R_1\equiv R_2\equiv R_3$ is also included) and $R_1(a,b)\in V$;

\item add \texttt{rch}$(R_1(a,b)$, $R_3(a,c))$ to $S_{\textit{\!\tiny rch}}$ where $R_1(a,b),R_2(b,c)\rightarrow R_3(a,c)\in P^*$,
    $R_1\circ R_2\sqsubseteq R_3\in\mathcal{R}$ and $R_2(b,c)\in V$.\hfill$\Box$
\end{enumerate}

Based on Algorithm~$\mathsf{PRC}$, we modify Algorithm~$\mathsf{A}_{\text{opt}}$
by replacing Step~\ref{rch} of Algorithm~$\mathsf{OPT}$
with Algorithm~$\mathsf{PRC}$. Algorithm~$\mathsf{A}_{\text{prc}}$
is, thus, given as follows:\\

\noindent\texttt{Algorithm~$\mathsf{A}_{\text{prc}}$}. Given
a DHL$(\circ)$ ontology $\mathcal{O}=\langle\mathcal{T},\mathcal{R},\mathcal{A}\rangle$ and its
datalog program $P=\langle R, \textbf{I}\rangle$, this algorithm
returns a materialization graph $\mathcal{G}$ of $P$.
Initially $\mathcal{G}$ is empty. The following steps are then performed:
\begin{enumerate}[leftmargin=8ex,label=(\textit{Step \arabic*}),ref=Step~\arabic*]
\item Add all facts in $\textbf{I}$ to $\mathcal{G}$.\label{alg4:addFacts}
\item Compute $S_{\textit{\!\tiny rch}}$ by performing Algorithm~$\mathsf{PRC}$; use an \texttt{NC}
    algorithm to compute the transitive closure $S^*_{\textit{\tiny rch}}$ (see \ref{transClos} in Algorithm~$\mathsf{OPT}$);
    update $\mathcal{G}$ by performing \ref{updateG} in Algorithm~$\mathsf{OPT}$.\label{alg4:updateG}
\item If no node has been added to $\mathcal{G}$ (in \ref{alg4:updateG}), terminate,
    otherwise iterate \ref{alg4:updateG}. \label{alg4:halt}\hfill$\Box$
\end{enumerate}

\begin{theorem}\label{theorem:aprc}
For any DHL$(\circ)$ ontology $\mathcal{O}$, Algorithm~$\mathsf{A}_{\text{prc}}$ halts and outputs
a materialization graph of $\mathcal{O}$.
\end{theorem}

The above theorem is given to show the correctness of Algorithm~$\mathsf{A}_{\text{prc}}$.
Since Algorithm~$\mathsf{A}_{\text{prc}}$ is almost identical with Algorithm~$\mathsf{A}_{\text{opt}}$,
it has the same upper bound of the computing time, i.e., $c_1+\ell'c_3 + \ell'c_4\log(|\textbf{I}|e) +\ell'c_5\log^2(|\textbf{I}|e)$,
where $\ell'$ denotes the number of iterations of \ref{alg4:updateG}, and the symbols
$c_1, c_3, c_4, c_5$ have the same meanings as those in the analysis of Algorithm~$\mathsf{A}_{\text{opt}}$.
The difference between the two algorithms is that Algorithm~$\mathsf{A}_{\text{prc}}$
considers a smaller scope of rule instantiations. This gives rise to smaller sizes of
the relation $S_{\textit{\!\tiny rch}}$. Based on the above discussion and Algorithm~$\mathsf{PRC}$,
the size of relation $S_{\textit{\!\tiny rch}}$ is bounded by
$h(|\textbf{I}|e)$ (where $h$
denotes the number of nodes that have been added to the graph), while
the size of the relation $S_{\textit{\!\tiny rch}}$ as computed by Algorithm~$\mathsf{A}_{\text{opt}}$
is at most $p^2(|\textbf{I}|e)^{2w}$.

We next use an example to show how Algorithm~$\mathsf{A}_{\text{prc}}$ handles
DHL$(\circ)$ materialization.

\begin{example}\label{exp:prc}
Consider performing Algorithm~$\mathsf{A}_{\text{prc}}$ on the ontology
$\mathcal{O}_{ex_1}$ of Example~\ref{exp:mg}. Note that the individual
set \textbf{IN} is $\{a_1,...,a_k,b\}$, i.e., $k+1$ individuals are involved.
Initially, Algorithm~$\mathsf{A}_{\text{prc}}$ adds
all the facts ($A(b),R(a_1,b),S(a_2,a_1),...,S(a_{k},a_{k-1})$) to the
result $\mathcal{G}_{ex_1}$ (\ref{alg4:addFacts}). In the first iteration of
\ref{alg4:updateG}, Algorithm~$\mathsf{A}_{\text{prc}}$ computes $S_{\textit{\!\tiny rch}}$ first.
According to Algorithm~$\mathsf{PRC}$, for each atom of the form
$S(a_i,a_{i-1})$, $2\leq i\leq k$,
and $\forall o\in \mbox{\textbf{IN}}$,
an $\texttt{\emph{rch}}$ relation of the form $\texttt{\emph{rch}}(R(a_{i-1},o),R(a_i,o))$ is added to
$S_{\textit{\!\tiny rch}}$. Since the atom $R(a_1,b)$ has been added to $\mathcal{G}_{ex_1}$,
Algorithm~$\mathsf{A}_{\text{prc}}$ checks that all atoms of the form
$R(a_i,b)$, $2\leq i\leq k$,
have SWD paths and are added to $\mathcal{G}_{ex_1}$ in the first iteration.
In the second iteration of \ref{alg4:updateG}, since all atoms of the
form $R(a_i,b)$, $1\leq i\leq k$,
have been in $\mathcal{G}_{ex_1}$, the $\texttt{\emph{rch}}$ relations of the form $\texttt{\emph{rch}}(A(b),A(a_i))$
are checked with respect to the axiom $\exists R.A\sqsubseteq A$; further all
atoms of the form $A(a_i)$, $1\leq i\leq k$, are finally added to $\mathcal{G}_{ex_1}$
by Algorithm~$\mathsf{A}_{\text{prc}}$.
\end{example}

From the above example, one can find that Algorithm~$\mathsf{A}_{\text{prc}}$ terminates after two iterations
of \ref{alg4:updateG}. This is the same as Algorithm~$\mathsf{A}_{\text{opt}}$ (see Example~\ref{exp:opt}).
We use Theorem~\ref{theorem:aprcpt} to show that Algorithm~$\mathsf{A}_{\text{prc}}$ also has an \texttt{NC} version when handling ontologies
that satisfy the simple-concept and the simple-role restrictions.
The correctness of this theorem is based on the fact that the method of narrowing down the scope of rule instantiations in Algorithm~$\mathsf{A}_{\text{prc}}$
does not influence the determination of the existence of SWD paths. A detailed analysis
can be found in the proof in the appendix.

\begin{theorem}\label{theorem:aprcpt}
For any DHL$(\circ)$ ontology $\mathcal{O}$ that satisfies the simple-concept and the simple-role
restriction,
there exists a poly-logarithmically bounded function $\psi$,
such that Algorithm~$\mathsf{A}_{\text{prc}}^{\psi}$ outputs
a materialization graph of $\mathcal{O}$.
\end{theorem}




%%% Local Variables:
%%% mode: latex
%%% TeX-master: "parallel-tractability-J"
%%% End:
