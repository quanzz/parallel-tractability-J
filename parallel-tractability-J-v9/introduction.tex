\section{Introduction}
\label{sec:introduction}

The Web Ontology Language (OWL)\footnote{The latest version is OWL 2: http://www.w3.org/TR/owl2-overview/}
is an important standard for ontology languages in the Semantic Web and knowledge-based applications.
In many of these applications, \emph{materialization} plays an important role, which is the reasoning task of computing all implicit
\emph{facts} that follow from a given ontology \cite{handbook}. Ontology developers and users employ materialization to optimize tasks such as query answering, ontology diagnosis or debugging. Since large amounts of semantic data
are being generated at an increasing pace by sensor networks, government authorities and social
media \cite{LehmbergRMB16,MeuselBP15}, it is challenging to conduct materialization on such large-scale ontologies efficiently.

For the ontology language RDFS and its extended fragments, approaches for parallel reasoning and for employing parallel computing platforms exist
 \cite{MotikNPHO14,PetersSZ15,SubercazeGCL16}. Several optimization strategies, e.g., dictionary encoding, balancing workload
and data partitioning, are further studied to enhance parallel RDFS reasoning. There are also parallel implementations for scalable
reasoning over ontologies that use highly expressive ontology languages \cite{SteigmillerLG14,WuH12}. The different approaches utilize different kinds of computing platforms to make reasoning tasks more efficient in parallel, e.g., supercomputers \cite{Hoeksema2011,GoodmanJMAAH11},
MapReduce \cite{UrbaniKMHB12} and GPU servers \cite{HeinoP12}.

The effectiveness of parallel reasoning for the above mentioned techniques is empirically verified on different test datasets.
However, materialization is not tractable in parallel for most of the popular ontology languages with \texttt{PTime}-complete or higher complexity \cite{Raymond95}. In particular, this is the case for RDFS and datalog rewritable ontology languages and, therefore, the efficiency of reasoning may not be
improved using a parallel implementation in theory. A possible reason for the apparent high performance of parallel
reasoning is that the utilized test datasets do not fall into the worst cases in terms of computational complexity. Also some well-known, large-scale ontologies such as YAGO show good performance for parallel
reasoning \cite{KolovskiWE10}, although they are expressed in ontology languages that are, in theory, not tractable in parallel.
The current theoretical work of parallel ontology reasoning can hardly explain this.
While one can try out different parallel implementations to see whether an ontology can be handled by (one of) them efficiently,
we study the problem of parallel tractability in theory and identify properties that make an ontology tractable in parallel.
These properties can further be used to optimize parallel algorithms and to guide ontology engineers in creating ontologies
for which parallel tractability can be guaranteed theoretically.

According to \citet{MotikNPHO14}, many real large-scale ontologies are essentially expressed in ontology languages
that can be rewritten into datalog rules. We follow this argumentation and focus on such datalog rewritable ontology languages in this paper.
The main target of this paper is to identify classes of datalog rewritable ontologies such that materialization
over these ontologies is tractable in parallel, i.e., falls into the parallel complexity class \texttt{NC} \cite{Raymond95}.
This complexity class consists of problems that can be solved efficiently in parallel.
To show that a problem is in \texttt{NC}, one can give an \emph{\texttt{NC} algorithm} that handles this problem
using parallel computation \cite{Raymond95}.
An \texttt{NC} algorithm is required to terminate in parallel poly-logarithmic time.
However, current materialization algorithms of datalog rewritable ontology languages
(e.g., the core algorithm used in RDFox \cite{MotikNPHO14}) are not \texttt{NC} algorithms
since they are designed for general datalog programs and have \texttt{PTime}-complete complexity.
Thus, we first give \texttt{NC} algorithms that perform materialization
and then identify the corresponding classes of datalog rewritable ontologies (called \emph{parallel tractability classes})
that can be handled by these \texttt{NC} algorithms. To make the proposed \texttt{NC} algorithms practical,
we also discuss how to optimize and implement them.

In the practical part of this work, we study specific datalog rewritable ontology languages.
We first focus on the ontology language DL-Lite \cite{CalvaneseGLLR07} to clarify how to apply the above \texttt{NC} algorithms
to study parallel tractability of ontology materialization.
We show that DL-Lite$_{\text{core}}$ and DL-Lite$_\mathcal{R}$ are tractable in parallel.
We then study the ontology language Description Horn Logic (DHL) \cite{GrosofHVD03}, which is the intersection of datalog and OWL
in terms of expressivity. We give a case of a DHL ontology where materialization can hardly be parallelized.
Based on the analysis of this case, we propose to restrict the usage of DHL such that materialization over the
restricted ontologies can be handled by the proposed \texttt{NC} algorithms.
We further extend the results to an extension of DHL that also allows complex role inclusion axioms.
Finally, we analyze the well-known benchmark, LUBM, and the real-world dataset, YAGO, and 
show that these ontologies following the
proposed restrictions belong to the parallel tractability classes.
We implement a system based on an optimized \texttt{NC} algorithm for DHL materialization
and show that this system is comparable to the state-of-the-art reasoner RDFox
when handling the LUBM and the YAGO ontologies.

The remainder of the paper is organized as follows. In Section~\ref{sec:background}, we introduce some basic notions.
We then give two \texttt{NC} algorithms for ontology materialization in Section~\ref{sec:ptclass}.
We study parallel tractability of materialization in DL-Lite, DHL and an extension of DHL in Section~\ref{sec:ptonto}, respectively.
In Section~\ref{sec:practicalAlg},
we discuss how to optimize and implement the given \texttt{NC} algorithms.
We analyze the LUBM and the YAGO ontologies and evaluate our implementation in Section~\ref{sec:evaluation}.
Finally, we discuss related work in Section~\ref{sec:related} and conclude in Section~\ref{sec:conclusion}. Detailed proofs of the theorems and lemmas in this paper are given in the appendix.



%%% Local Variables:
%%% mode: latex
%%% TeX-master: "parallel-tractability-J"
%%% End:
