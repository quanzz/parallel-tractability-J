\section{Introduction}
\label{sec:introduction}

The Web Ontology Language (OWL)\footnote{The latest version is OWL 2: http://www.w3.org/TR/owl2-overview/}
is an important standard for ontology language in the Semantic Web and knowledge-based applications.
In these applications, \emph{materialization}, which is the reasoning task of computing all implicit
facts that follow from a given ontology \cite{handbook}, plays an important role. By means of it, developers and users
can optimize query answering, ontology diagnosis and debugging. Since huge amounts of semantic data
are being generated at an increasing pace by sensor networks, government authorities and social
media \cite{LehmbergRMB16,MeuselBP15}, it is challenging to conduct materialization on such large-scale ontologies efficiently.

There has been work proposing parallel reasoning algorithms and employing parallel computing platforms
for the ontology language RDFS and its extended
fragments \cite{MotikNPHO14,PetersSZ15,SubercazeGCL16}. Several optimization strategies, e.g., dictionary encoding, balancing workload
and data partitioning, are further studied to enhance parallel RDFS reasoning. There are also parallel implementations for scalable
reasoning of highly expressive ontology languages \cite{SteigmillerLG14,WuH12}. On the other hand, several work utilizes different kinds of computing platforms to make reasoning tasks more efficient in parallel, such as supercomputers \cite{Hoeksema2011,GoodmanJMAAH11},
MapReduce \cite{UrbaniKMHB12} and GPU servers \cite{HeinoP12}.

The above work verifies the effectiveness of parallel reasoning based on the evaluations on different test datasets.
However, for most popular ontology languages, even RDFS and datalog rewritable ontology languages,
which have \texttt{PTime}-complete or higher complexity of reasoning in the worst case,
they are not parallelly tractable according to \cite{Raymond95}, i.e., the efficiency of reasoning may not be
improved on a parallel implementation. A possible reason of the high performance of parallel
reasoning is that the utilized test datasets do not fall in the worst cases in terms of computational complexity.
It has also been shown that some well-known large-scale ontologies, such as YAGO, have good performance for parallel
reasoning \cite{KolovskiWE10}, but they are expressed in ontology languages that are not parallelly tractable in theory.
On the other hand, there do exist ontologies for which materialization cannot
always be improved by parallelism (we use experiments to show this in Section~\ref{sec:evaluation}).
The current work of parallel ontology reasoning can hardly explain this.
While one can try out different parallel implementations to see whether an ontology can be handled by (one of) them efficiently,
we study the problem of parallel tractability in theory and identify properties that make an ontology parallelly tractable.
These properties can be further used to optimize parallel algorithms and to guide ontology engineers in creating ontologies
for which parallel tractability can be guaranteed theoretically.

According to \cite{MotikNPHO14}, many real large-scale ontologies are essentially expressed in the ontology languages
that can be rewritten into datalog rules. Thus, we focus on such datalog rewritable ontology languages in this paper.
The main target of this paper is to identify the classes of datalog rewritable ontologies such that materialization
over these ontologies is parallelly tractable, i.e., in the parallel complexity class NC \cite{Raymond95}.
This complexity class consists of problems that can be solved efficiently in parallel.
To show that a problem is in the NC class, one can give an \emph{NC algorithm} that handles this problem
in parallel computation \cite{Raymond95}.
An NC algorithm is required to terminate in parallel poly-logarithmic time.
However, current materialization algorithms of datalog rewritable ontology languages
(e.g., the core algorithm used in RDFox \cite{MotikNPHO14}) are not NC algorithms,
since they are designed for general datalog programs and has \texttt{PTime}-complete complexity.
Thus, we first give NC algorithms that perform materialization,
and then identify the corresponding classes of datalog rewritable ontologies (called \emph{parallelly tractable classes})
that can be handled by these NC algorithms. To make the proposed NC algorithms practical,
we also discuss how to optimize and implement them.

In the practical part of this work, we study specific datalog rewritable ontology languages.
We first focus on the ontology language DL-Lite to clarify how to apply the above NC algorithms
to study the parallel tractability of ontology materialization.
We show that DL-Lite$_{core}$ and DL-Lite$_\mathcal{R}$ are parallelly tractable.
We then study the ontology language Description Horn Logic (DHL) \cite{GrosofHVD03}, which is the intersection of datalog and OWL
in terms of expressivity. We give a case of a DHL ontology where materialization can hardly be parallelized.
Based on the analysis of this case, we propose to restrict the usage of DHL such that materialization over the
restricted ontologies can be handled by the proposed NC algorithms.
We further extend the results to an extension of DHL that also allows complex role inclusion axioms.
Finally, we analyze well-known benchmarks and real-world datasets and show that many real ontologies following the
proposed restrictions belong to the parallelly tractable classes.
We implement a system based on an optimized NC algorithm for DHL materialization.
We compare our system to the state-of-the-art reasoner RDFox.
The experimental results show that the optimization proposed in this paper results in a better performance on parallelly
tractable ontologies compared with RDFox.

The rest of the paper is organized as follows. In Section~\ref{sec:background}, we introduce some basic notions.
We then give two NC algorithms for ontology materialization in Section~\ref{sec:ptclass}.
We study the parallelly tractable materialization of DL-Lite, DHL and an extension of DHL in Section~\ref{sec:ptonto} respectively.
In Section~\ref{sec:practicalAlg},
we discuss how to optimize and implement the given NC algorithms.
We analyze real-world datasets and evaluate our implementation in Section~\ref{sec:evaluation}.
We then discuss related work in Section~\ref{sec:related} and conclude in Section~\ref{sec:conclusion}.
The detailed proofs of the theorems and the lemmas in this paper are given in the appendix.



%%% Local Variables:
%%% mode: latex
%%% TeX-master: "parallel-tractability-J"
%%% End:
