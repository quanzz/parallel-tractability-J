\documentclass{article}
\usepackage{amsmath}
\usepackage{url}

\let\quoteOld\quote
\let\endquoteOld\endquote
\renewenvironment{quote}{\quoteOld\itshape}{\endquoteOld}

% use the following command line to convert to standard text file
% pandoc reviews.tex -o rebuttal.txt
\setlength{\parindent}{0pt}

\usepackage[usenames, dvipsnames]{color}
\newcommand{\BG}[1]{\textcolor{red}{BG: #1}}
%\renewcommand{\BG}[1]{}

\begin{document}

We thank the reviewers for the time and effort invested in the revised
submission and their helpful comments to improve the paper. Below we
reply to the points raised by the reviewers.


\section{Rebuttal to Review 1}

\begin{quote}
In Section 4.4 it should be "can be derived" instead of "could be derived".
The wording in the submission looks as if that is something that is no
longer possible.
\end{quote}

done

\begin{quote}
RDF(S) materialization is also infinite because of the infinite non-trivial
vocabulary for RDF containers.  This doesn't matter in theory because the
inferences can be easily summarized and doesn't matter in practice because
this vocabulary is hardly used.  However, there should be a bit on it
anyway.
\end{quote}

We discuss this now in Sec.~4.4 on p.~17.


\section{Rebuttal to Review 2}

\begin{quote}
  For the revised Definition~1, I wonder why it requires all datalog
  programs in a class $\mathcal{D}$ to share the same rule set. And it
  contradicts to the claim after the definition that each \texttt{NC}
  algorithm $\mathsf{A}$ corresponds to a class
  $\mathcal{D}_\mathsf{A}$. Does it mean a \texttt{NC} algorithm only
  applies to one rule set? To study data complexity, one can assume
  the size of the rule set is fixed, but I do not see why the rules
  need to be fixed.
\end{quote}

Since we study the data complexity (see last sentence before
Section~2.4 and the second sentence of Sec.~3), we consider only the
data (the facts $\textbf{I}$ as input), whereas the rule set is
assumed to be fixed. This is the standard definition for data
complexity. We have rephrased Definition~1 and now use a phrasing
closer to that used by Lutz and Wolter \cite{LuWo17a}, which hopefully
makes this clearer. It is important (and not specific to our work)
that only the data is considered as (variable) input.

This does not mean that an \texttt{NC} algorithm applies to just one
rule set, but it means that for studying the data complexity, we
assume that only the input data can be different (is in the input),
whereas the rule set is considered fixed.

Hence, while the algorithms (e.g.\ Algorithm $\mathsf{A}_\text{bsc}$
(or $\mathsf{A}_\text{opt}$) involve two inputs (rules and facts),
for the complexity analysis, we assume the rules part of the input to
be fixed. In general, however, the algorithms can handle different
rule sets. Please note that we still have to show that the (data)
complexity holds for all possible sets of facts.

\begin{quote}
  The revised Definition~2 also has some issues. The second condition
  does not preclude the cases where original facts get parents and
  where derived facts get no parents; and the third condition does not
  mention which rules in $P^*$ the graph $G$ must satisfy or wether
  $G$ need to satisfy any rule at all.
\end{quote}

Thanks for pointing that out. The definition indeed had issues and we
have revised it. Original facts now have to be present in $G$ and have
to have an in-degree of $0$. Derived facts now have to have parents
(otherwise they would have an in-degree of $0$, which is only allowed
for original facts) and the parents correspond to a ground rule that
allows for deriving the fact. We have also moved the definition of a
\emph{complete} materialization graph forward, which was so far in
text and not in the definition environment. This will hopefully make
it clearer that not all derivable facts have to be present in a
materialization graph unless it is a complete materialization graph.

We have further added a simple example (now Example~1) to illustrate
the different aspects of the definition. 

\begin{quote}
  In Algorithm $\mathsf{A}_\text{bst}$, the reason why Step~2 costs
  constant time is still unclear. The paragraph below Lemma~1 does not
  provide any explanation. The authors pointed out two references in
  their response, but I could not find any clue in those articles (one
  of which is just a two page summary \textcolor{red}{I can also not
    see why this reference is of any help. Maybe we should comment on
    this? }). \textcolor{red}{For the second reference
    \cite{Raymond95}, I can also not see where it talks about matrix
    multiplication? Does this follow from what is written below
    Def.~3.2.10?} Again, even under the assumption that there is an
  one-to-one correspondence between processors and rule instances, for
  each processor and the corresponding rule $B_1, \ldots, B_n \to H$,
  the complexity of checking the applicability of
  $B_1, \ldots, B_n \to H$ in $G$ and at the same time updating $G$
  should depend on the size of $G$.
\end{quote}

One issue might have been our use of the two terms `rule instance'
and `ground instantiation of a rule', which denote the same thing (but
this was maybe not clear in the paper).
We now unify the terms and only use `rule instantiation', which is introduced
in Sec.~2.1. A rule instantiation is different from a rule.
A rule instantiation has no variable while a rule can have variables.
Checking the applicability of a rule depends on the size of the
processed graph. However, checking the applicability of a rule
instantiation only depends on the derivability of the ground atoms
in this rule instantiation. Suppose that a processor can access the
location of any atom in constant time. It can also check the applicability of
a rule \emph{instantiation} in constant time.

The two references given in the previous response were pointing to two
other kinds of problems, i.e., the computation of the transitive
closure and matrix manipulation (see
\cite[Part~\uppercase\expandafter{\romannumeral2}~A]{Raymond95}, to
show that the constant time assumption is also
applicable. Unfortunately, we referred to `matrix manipulation' as
`matrix multiplication' in the previous response by accident. The
first reference in the previous response gives related works about
this issue without covering the details. The second reference is
indeed too brief to go much beyond the problem description. In the
following, we give a more detailed explanation.

For the constant time claims, an important precondition is the
computation model used for analysis. Parallel random access machine
(PRAM for short) is used as the basic computation model in this
paper. PRAM is an extension of the random access machine (RAM) model.

1) Suppose that there is a one-to-one correspondence between
processors and rule instantiations, and each processor can access the
location of any atom \textcolor{red}{within a rule? atom here means
  derived facts?} in constant time
(given as the parallel assumption in Section~3.3). Since a processor
$p$ corresponds to a rule instantiation, e.g.,
$B_1, \ldots, B_n \to H$, processor $p$ knowns the relationship among
$B_1, \ldots, B_n$ and $H$. Further, processor $p$ knows the locations
of the nodes $B_1, \ldots, B_n$ and $H$ \textcolor{red}{within the
  materialization graph? how?  by checking (in constant time) the
  derived facts?}. Thus, it could check the
derivability of $B_1, \ldots, B_n$ \textcolor{red}{Check the
  derivability or whether the (ground) atoms have been derived?}, and add an edge from $B_i$ to $H$
($1 \leq i \leq n$) in constant time instead of searching the whole
graph $G$.

2) How to establish such a one-to-one correspondence between processors
and rule instantiations and the constant-time access to each atom for processors?

We should also introduce the notion of Boolean circuits, which are 
circuits that consist of logic gates, e.g., AND and OR gates (see
\cite[Section~2.3]{Raymond95}).  A materialization graph can also be
seen as a Boolean circuit by treating the nodes as AND gates.\\
%
First, the evaluation of a Datalog program is equivalent to the evaluation
of its corresponding Boolean circuit in terms of computational
complexity \cite{DMRT14a}. \\
%
Second, a Boolean circuit can be simulated by PRAM based on an
algorithm called parallel prefix computation \cite{LaFi80a}.  Further,
the complexity of simulation is lower than that of the evaluation of
the Boolean circuit \cite[Lemma~2.4.1]{Raymond95}.\\
%
Third, during the above simulation procedure, a one-to-one
correspondence between each gate in the Boolean circuit and the
processors in PRAM and the constant-time access are established.\\
%
In our case, the one-to-one correspondence between processors and rule
instantiations and the constant-time access to each node can also be
established by PRAM similarly to the above simulation procedure.

\textcolor{red}{We have added the above argumentation also to the
  paper. (needed?, TBD) \emph{Rephrase the relevant paragraphs in a brief way?}}

\begin{quote}
  Again, in Algorithm $\mathsf{A}_\text{opt}$ the claim that the
  computation of $S_\text{rch}$ and updating $G$ cost only constant
  time is unclear.
\end{quote}

\textcolor{red}{The argumentation for this claim is similar to the one
  given above. (?) Is it? Can we simply reply like that?}

\bibliographystyle{elsarticle-num}
\bibliography{response2}

\end{document}

%%% Local Variables:
%%% mode: latex
%%% TeX-master: t
%%% End:
