\section{Background Knowledge}
\label{sec:background}

In this section, we introduce some basic notions that are needed to introduce our approach.


\subsection{Datalog}

We discuss the main issues in this paper using standard datalog notions.
In datalog \cite{database}, a \emph{term} is a variable or a constant. An \emph{atom} $A$
is defined by $A\equiv p(t_1,\ldots,t_n)$ where $p$ is a \emph{predicate} (or \emph{relational})
name, $t_1,\ldots,t_n$ are terms, and $n$ is the arity of $p$. If all the terms in an atom $A$ are
constants, then $A$ is called a \emph{ground atom}.
A datalog \emph{rule} is of the form: `$B_1,\ldots,B_n\rightarrow H$',\footnote{In datalog rules, a comma
represents a Boolean conjunction `$\wedge$'.} where $H$ is referred to as
the \emph{head atom} and $B_1,\ldots,B_n$ the \emph{body atoms}. Each variable in the head atom
of a rule must occur in at least one body atom of the same rule. A \emph{fact} is a rule of
the form `$\rightarrow H$', i.e., a rule with an empty body and the head $H$ being a ground atom.
A datalog program $P$ consists of rules and facts.
A \emph{substitution} $\theta$ is a partial mapping of variables to constants.
For an atom $A$, $A\theta$ is the result of replacing each variable $x$ in $A$
with $\theta(x)$ if the latter is defined. We call $\theta$ a \emph{ground substitution}
if each defined $A\theta$ is a ground atom.
A \emph{ground instantiation} of a rule is obtained by applying a ground substitution on all
the terms in this rule with respect to a finite set of constants occurring in $P$.
Furthermore the ground instantiation of $P$, denoted by $P^*$,
consists of all ground instantiations of rules in $P$.
The predicates occurring only in the body  of some rules are called \emph{EDB predicates},
while the predicates that may occur as head atoms are called \emph{IDB predicates}.


\subsection{DHL and DL-Lite}

In what follows, we use $\textbf{CN}$, $\textbf{RN}$ and $\textbf{IN}$
to denote three disjoint countably
infinite sets of \emph{concept names, role names}, and \emph{individual names} respectively.
The set of roles is defined as $\textbf{R}:=\textbf{RN}\cup\{R^-|R\in\textbf{RN}\}$
where $R^-$ is the \emph{inverse role} of $R$.
For ease of discussion, we focus on the \emph{simple forms} of axioms shown
in the left column of Table~\ref{tab:dhl}. These simple forms can be obtained by using
well-known \emph{structural transformation} techniques \cite{KrotzschRH07,Kazakov09}.

DHL (short for \emph{description horn logic}) \cite{GrosofHVD03} is introduced as an
intersection of description logic (DL) and datalog in terms of expressivity.
We define a DHL ontology $\mathcal{O}$ as a triple:
$\mathcal{O}=\langle\mathcal{T},\mathcal{R},\mathcal{A}\rangle$, where
$\mathcal{T}$ denotes the TBox containing axioms of the forms (T1-T3);
$\mathcal{R}$ is the RBox that is a set of axioms of the forms (R1-R3);
$\mathcal{A}$ is the ABox containing \emph{assertions} of the forms (A1) and (A2).
In an axiom of either of  the forms (T1-T3 and R1-R3), concepts $A_{(i)}$ and $B$ are either
concept names, the \emph{top concept} ($\top$) or the \emph{bottom concept} ($\bot$); $R$ and $S_{(i)}$
are roles in $\textbf{R}$.
For an axiom of the form $A\sqsubseteq\forall R.B$ that is also allowed in DHL, we only consider its
equivalent form $\exists R^-.A\sqsubseteq B$.

DHL is related with other ontology languages.
First, DHL is essentially a fragment of the description logic Horn-$\mathcal{SHOIQ}$ with
disallowing \emph{nominals}, \emph{number restrictions} and
right-hand side \emph{existential restrictions} ($A\sqsubseteq\exists R.B$).
Second, the expressivity of DHL covers that of RDFS to some extent \cite{GrosofHVD03}.
Reasoning with RDFS ontologies is \texttt{NP}-complete \cite{Horst05}
and, thus, is not tractable in parallel.
However, by applying some simplifications and restrictions, RDFS ontologies can be
expressed in DHL \cite{GrosofHVD03}, which has \texttt{PTime}-complete complexity for
materialization.

\begin{table}
\centering
\caption{Axioms and corresponding datalog rules}
\begin{tabular}{l >{\hspace*{12mm}}c<{\hspace*{12mm}} >{\hspace*{12mm}}c<{\hspace*{12mm}}}
\hline
& Axioms & Datalog Rules\\
\hline
\hline
(T1)& $A\sqsubseteq B$& $A(x)\rightarrow B(x)$\\
(T2)& $A_1\sqcap A_2\sqsubseteq B$& $A_1(x), A_2(x)\rightarrow B(x)$\\
(T3)& $\exists R.A\sqsubseteq B$& $R(x,y), A(y)\rightarrow B(x)$\\
(T4)& $A\sqsubseteq\exists R$& $A(x)\rightarrow R(x, o_{R}^A)$\\

\hline
(R1)& $S\sqsubseteq R$& $S(x,y)\rightarrow R(x,y)$\\
(R2)& $S\sqsubseteq R^-$& $S(x,y)\rightarrow R(y,x)$\\
(R3)& $R\circ R\sqsubseteq R$& $R(x,y), R(y,z)\rightarrow R(x,z)$\\
(R4)& $R_1\circ R_2\sqsubseteq R$&$R_1(x,y),R_2(y,z)\rightarrow R(x,z)$\\

\hline
(A1)& $A(a)$& $A(a)$\\
(A2)& $R(a,b)$& $R(a,b)$\\
\hline
\end{tabular}
\label{tab:dhl}
\end{table}

In the initial work of DHL \cite{GrosofHVD03}, \emph{complex role inclusion axioms} (complex RIAs) of
the form $R_1\circ\ldots\circ R_n\sqsubseteq R$ are not considered, although they can be
naturally transformed into datalog rules.
In this paper, we also consider an extension of DHL (denoted by DHL($\circ$))
that allows complex RIAs. Since a complex RIA can be transformed to
several axioms of form (R4), we then require that an
RBox $\mathcal{R}$ of a DHL($\circ$) ontology can contain
axioms of the forms (R1-R4). Note that (R3) is actually a special
case of (R4).

DL-Lite is a group of ontology languages designed for highly-efficient
query answering over knowledge bases
and underpins the OWL profile OWL QL \cite{CalvaneseGLLR07}.
DL-Lite$_{\text{core}}$ and DL-Lite$_{\mathcal{R}}$ are the two basic fragments
of DL-Lite.
DL-Lite$_{\text{core}}$ requires that a TBox only contains axioms of the forms (T1) $A\sqsubseteq B$,
(T2) $A_1\sqcap A_2\sqsubseteq B$ where $B\equiv\bot$, (T3) $\exists R.A\sqsubseteq B$
where $A\equiv\top$ or $B\equiv\bot$, and (T4) $A\sqsubseteq\exists R$,
an ABox contains \emph{assertions} of the forms (A1) and (A2), and
RBoxes are not allowed. DL-Lite$_{\mathcal{R}}$ is an extension of DL-Lite$_{\text{core}}$,
which allows involving RBoxes that contain axioms of the forms (R1-R2).
If not specially specified, \emph{a DL-Lite ontology denotes a DL-Lite$_{\mathcal{R}}$ ontology
in the following paragraphs}.


\subsection{Ontology Materialization via Datalog Programs}

An ontology
expressed in DHL, DHL($\circ$), DL-Lite$_{\text{core}}$ or DL-Lite$_{\mathcal{R}}$ can be transformed into a datalog program
(see the corresponding rules in the right column of Table~\ref{tab:dhl}).
Note that an axiom of form (T4) $A\sqsubseteq\exists R$ should be
transformed into a
first-order logic rule of the form $A(x)\rightarrow \exists y(R(x,y))$, where $y$ is called a \emph{free variable} (it is
not occurring in the body atom $A(x)$).
In this work, such a free variable $y$ is eliminated via Skolemization into a fresh individual $o_{R}^A$
that corresponds to the concept $A$ and the role $R$. The elimination of free variables
does not influence the result of materialization \cite{CalvaneseGLLR07}.

In what follows, for an ontology $\mathcal{O}=\langle\mathcal{T},\mathcal{R},\mathcal{A}\rangle$,
we use $P=\langle R, \textbf{I}\rangle$ to represent the corresponding datalog program
where $R$ is the set of rules obtained by transforming the axioms in
$\mathcal{T}$ and $\mathcal{R}$ and
$\textbf{I}$ is the set of facts that are directly copied from the assertions in $\mathcal{A}$.
Further, we use $R_1\sqsubseteq_{*}R_2$ to denote the smallest transitive reflexive relation
between roles such that $R_1\sqsubseteq R_2\in\mathcal{R}$ implies $R_1\sqsubseteq_{*}R_2$
and $R_1^-\sqsubseteq_{*}R_2^-$. We omit the reference to
$\mathcal{R}$ if it is clear from the context. In this paper, we also use the
notion of \emph{simple role}, which was initially proposed to restrict the
usage of highly expressive ontology languages \cite{HorrocksS04}.
Specifically, a role $S\in\textbf{R}$ is \emph{simple} if, (1) it has no subrole (including $S$)
occurring on the right-hand side of axioms of the forms (R3) and (R4); (2) $S^-$ is simple.

Based on the above representations, ontology materialization
corresponds to the evaluation of datalog programs.
Specifically, given a datalog program $P=\langle R, \textbf{I}\rangle$,
let $T_R(\textbf{I})=\{H\theta|\forall B_1,\ldots,B_n\rightarrow H\in R,
B_i\theta\in\textbf{I}, 1\leq i\leq n \}$,
where $\theta$ is some substitution; further let $T_R^{0}(\textbf{I})=\textbf{I}$ and
 $T_R^{i}(\textbf{I})=T_R^{i-1}(\textbf{I})\cup T_R(T_R^{i-1}(\textbf{I}))$ for each $i>0$.
The smallest integer $n$ such that $T_R^{n}(\textbf{I})= T_R^{n+1}(\textbf{I})$ is called \emph{stage},
and \emph{materialization} refers to the computation of $T_R^{n}(\textbf{I})$ with respect to $R$ and \textbf{I}.
$T_R^{n}(\textbf{I})$ is also called the \emph{fixpoint} and denoted by $T_R^{\omega}(\textbf{I})$.
We say that an atom $A$ is \emph{derivable} or can be \emph{derived} with respect
to the datalog program $P=\langle R, \textbf{I}\rangle$ if $A\in T_R^{\omega}(\textbf{I})$.
In this paper, we consider the data complexity of materialization, i.e., we \emph{assume that the rule set $R$ is fixed}
for any class of datalog programs.


\subsection{The Complexity Class \texttt{NC}}

The parallel complexity class \texttt{NC}, known as Nick's
Class \cite{Raymond95}, is studied by theorists as a parallel complexity class
where each decision problem can be efficiently solved in parallel.
Specifically, a decision problem in the \texttt{NC} class
can be solved in poly-logarithmic time on a PRAM (parallel, random-access machine) with
a polynomial number of processors. We also say that an \texttt{NC} problem can be solved
in \emph{parallel poly-logarithmic time}.
Although the \texttt{NC} complexity class is a theoretical analysis tool,
it has been shown that many \texttt{NC} problems can be solved efficiently in practice \cite{Raymond95}.

From the perspective of implementations, \texttt{NC} problems are also
highly feasible in parallel for other parallel models like BSP \cite{Valiant90}
and MapReduce \cite{KarloffSV10}. The \texttt{NC} complexity class was originally defined
as a class of decision problems. Since we study the problem of materialization, we do not
require in this work that a problem is a decision problem in \texttt{NC}.
In addition, since many parallel reasoning systems (see related work in Section~\ref{sec:related})
are implemented on shared-memory platforms, we
study all the issues in this work by assuming that the running machines are in
shared-memory configurations.



%%% Local Variables:
%%% mode: latex
%%% TeX-master: "parallel-tractability-J"
%%% End:
