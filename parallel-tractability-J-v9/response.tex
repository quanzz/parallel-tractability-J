\documentclass{article}
\usepackage{amsmath}
\usepackage{url}

\let\quoteOld\quote
\let\endquoteOld\endquote
\renewenvironment{quote}{\quoteOld\itshape}{\endquoteOld}

% use the following command line to convert to standard text file
% pandoc reviews.tex -o rebuttal.txt
\setlength{\parindent}{0pt}

\usepackage[usenames, dvipsnames]{color}
\newcommand{\BG}[1]{\textcolor{red}{BG: #1}}
%\renewcommand{\BG}[1]{}

\begin{document}

We thank the reviewers for the time and effort invested in the
detailed reviews and the valuable suggestions and comments to improve
the quality of the paper. The paper has been carefully proof-read and,
hopefully, all wording and structural issues are fixed now. We also
fixed all mentioned minor issues. Below we reply to the other points
raised by the reviewers for which we completely revised the evaluation
section.


\section{Rebuttal to Review 1}

\begin{quote}
There should, however, be some discussion somewhere in the paper on the
relationship between path length and KB size.   Even if a KB uses a
non-parallelizable ontology language this doesn't matter much if the paths
that cause problems are short compared to the size of the KB.  The
non-parallel aspects of path processing of short paths will not be noticable
with a small number of threads (certainly up to hundreds) and maybe not even
for very many threads (because of general overhead).
\end{quote}

We added a parallel complexity analysis for Algorithm~$\mathsf{A}_{\text{bsc}}$
and Algorithm~$\mathsf{A}_{\text{opt}}$
in Section~3.3 and Section~3.4 respectively.
Based on the analysis and Theorem~1, we have that the parallel complexity
of Algorithm~$\mathsf{A}_{\text{bsc}}$ depends only on the graph depth,
while the parallel complexity of Algorithm~$\mathsf{A}_{\text{opt}}$
depends on both of the graph depth and the size of the input ontology.\\

We added an evaluation and analysis of the role of the graph depth in Section~6.3.
The experimental results show that, when the graph depth is comparable to
the size of the given ontology, it determines the reasoning time.


\begin{quote}
  The second sentence of Section 3 is badly stated and misleading.  It is definitely true that for a Datalog program $\langle R, I \rangle$ the rule set $R$ is fixed.  However, the initial facts $I$ are also fixed.
\end{quote}

We have rephrased the beginning of Section~3 and made it clear that the rule set is fixed for a class of Datalog programs.
We also revised the similar presentations in Section~2.

\begin{quote}
It is not clear whether the classes of Datalog programs considered in the
paper have to share the same rule set.  The discussion previous to
Definition 1 seems to so indicate, but there is nothing in the definition
that says so.
\end{quote}

We modified Definition~1 to make it clear that the Datalog programs
in a class share the same rule set.

\begin{quote}
There is no definition of what the parents of a node in a graph are.
However, it is quite obvious what the definition should be.
\end{quote}

We added the definition of the parents of a node in a materialization graph in Definition~2.

\begin{quote}
Example 6 does need a bit of revision.  It is not the case that Oex3 cannot
be materialzed tractably in parallel.  Instead there has to be a class of
ontologies that cannot be so materialized.
\end{quote}

We revised Example~6 (see the paragraph following Example~6) and also
and Example~1 (see the last paragraph of Section~3.2).


\begin{quote}
It would be much better if the ontology names corresponded to the example
numbers, i.e., the ontology in Example 6 was Oex6.
\end{quote}

We made the ontology subscripts correspond to the example numbers
for Example~5, Example~6, Example~7 and Example~9.


\begin{quote}
The notion of a simple concept is very much tied to the algorithm developed
in the paper.  How often do actual ontologies have constructs that could
make them intractable that are rended tractable by having simple concepts?
If there are none, then this class is uninteresting, particularly as the
paper makes such strong claims about worrying about particular ontologies.
This is different from an ontology only containing simple concepts.
\end{quote}


We investigated many real ontologies from different data source. These ontologies cover
several domains and only contain simple concepts (see
Section~6.1). \BG{Sec.~6.1 now only covers LUBM and YAGO. I don't think the
current answer is convincing. Did this refer to the old Sec.~6.1 with
the many more ontologies?}\\

In this work, we focused on exploring the properties
that make ontologies tractable in parallel. Thus, we did not study how to make ontologies
tractable in parallel by having simple concepts or making concepts
simple. This could be a direction for future work.


\begin{quote}
The paper is all about parallel tractability of materialization, both
theoretical and practical.  However the paper gives insufficient attention
to RDFS, which is the main ontology language used in practice, and
particularly for large knowledge bases.  Even though RDFS is laughably
simple it does not have parallel tractability of even entailment.  The paper
should better examine parallel tractability in the context of RDFS or RDFS
with minor extensions.

Of course, RDFS does not fit into the approach of the paper, as it does not
a-priori separate classes and individuals.  The paper needs to discuss this
issue, as it prominently mentions RDFS.
\end{quote}

We used a new section (Section~4.4) in the revision to discuss why RDFS
does not have parallel tractability of entailment.

\begin{quote}
The paper does a bad job of describing why materialization in RDFS is not
parallizable.  RDFS KBs have an infinite number of consequences, so it is
certainly not possible to materialize all of them quickly.  There are,
however, well known tricks that can be used to generate a finite
representation of this infinite set of consequences.  However, even this
cannot be done quickly in parallel.  In fact, even RDFS entailment cannot be
done quickly in parallel, because RDFS allows for a kind of twisted
interaction (which I think it similar to twisted paths in the paper) when
computing subproperty relationships.  Of course, having subproperties of
rdfs:subPropertyOf is not a usual thing to do in RDFS.  The paper needs to
be much more clear on its relationship to RDFS.
\end{quote}

We gave an example that RDFS entailment may lead to the situation of path twisting in Section~4.4 of the revision.
We also discussed in what cases RDFS ontologies are tractable in parallel.

\begin{quote}
The paper mentions YAGO early on but does not indicate why YAGO admits good
performance for parallel reasoning.  YAGO uses a very simple ontology
language - RDFS plus two simple extensions.  It is thus a prime candidate
for close examination to show how practical results differ from worst-case
theoretical ones.  However, the paper does not have any examination of
reasoning on YAGO.  Such an examination is very much called for.  Similarly
an examination of the performance of other large RDFS-based KBs, such as
DBpedia, would be very useful.   If YAGO is not benchmarked then it needs to
be dropped entirely.   Similarly for LUBM.
\end{quote}

We extended Section~6.1 to further discuss in which cases
the two ontologies, YAGO and LUBM, are tractable in parallel.
We added an evaluation for these ontologies in Section~6.2.

\begin{quote}
The benchmarking part of the paper is very limited and extremely hard to
follow.    Only eight ontologies were benchmarked.  Why were these eight
chosen?  Only considering at most eight threads is much too constricting.
\end{quote}

The eight ontologies were selected since they satisfy the properties for parallel tractability.
We used them to examine the effects of parallel tractability.\\

In the revision, we did not continue to use the eight ontologies.
We modified the LUBM ontology to generate ontologies of different graph depths.
The experimental results were clearer for explaining the issues of graph depths
and parallel tractability.\\

We used up to 24 threads in the revised experiments.


\begin{quote}
More detail is needed on how the KBs were constructed.  Chain length is a
very important aspect of the KBs, but nothing is said of how the instances
in the KB are connected together.  It is not stated which ontologies belong
to which DL.
\end{quote}

In the original experiments, we did not
construct these eight ontologies for a certain purpose, e.g., to
make the graph depth deeper. The main purpose was to closely reflect real-world scenarios. \\

In the revision, we modified the LUBM benchmark to generate ontologies of
different graph depths. We further provide a description of how these ontologies
were constructed.


\begin{quote}
Figure 5 is very misleading because it uses different scales for the two
systems.   There are several immediate questions that arise from this Figure
that are poorly handled in the paper.  First, why does going from one thread
to two make such a difference for ParallelDHL?  Without a convincing
explanation for this surprising aspect of the benchmarking it is hard to
trust any of the rest of the benchmarking.  The other surprising aspect of
Figure 5 is that RDFox speeds up so little.  There is a short explanation of
why this is so, but I would have liked to see a longer one.
\end{quote}


In the original experiments, ParallelDHL required a large amount of time for computing
the relation $S_{\textit{\!\tiny rch}}$ when only one thread is being allocated.
With several threads allocated, ParallelDHL has a better efficiency.
This also resulted in the difference of the speedups between RDFox and ParallelDHL.


\begin{quote}
The computation of the speedup numbers is wholely unexplained.  Why should
this number have any importance?
\end{quote}


We used the speedup numbers to show different performances between the
ontologies in $\mathcal{D}_{\textit{\text{dhl}}(\circ)}$ and
that not in $\mathcal{D}_{\textit{\text{dhl}}(\circ)}$.\\

In the revision, we provide the formula for the speedup and analyze
the speedup numbers.


\begin{quote}
There is a claim that the benchmarking distinguishes between the Ddhlo and
D-dhlo.  However, Figure 5 doesn't really show this well, if at all.   For
example, why should processing slow down with more threads at all?  Further,
it appears that even for several of the more well-behaved ontologies that there
is very little speedup between 6 and 8 processors.
\end{quote}

\BG{The answer is not very much to the point and just a repetition of
  a previous answer.}
In order to make the original experiments close to real-world scenarios, we did not
construct the eight ontologies for some special purpose, e.g., to
make the graph depth deeper.
The differences between the ontologies in
$\mathcal{D}_{\textit{\text{dhl}}(\circ)}$ and those not in
$\mathcal{D}_{\textit{\text{dhl}}(\circ)}$ were not clear
for the test ontologies.\\

In the revision, we did not continue to use the eight ontologies.
We modified the LUBM ontology to generate ontologies of different graph depths.
The experimental results were clearer for explaining the issues of graph depths
and parallel tractability.


\begin{quote}
The benchmarking should be redone with more ontologies and with more
threads.  As well, there needs to be better analysis of where parallelism is
failing.  For example, if chains are short, then there should be no problem
in achieving near-perfect thread utilization, ignoring memory contention
problems.
\end{quote}


We added a detailed analysis for LUBM and YAGO and the evaluation of
graph depths in the revision.


\begin{quote}
"parallely tractable" is very grating.  It would be better to rewrite
sentences to not use it.
\end{quote}

We have removed all occurrences of ``parallely tractable'' and
rephrased the sentenced to use ``tractable in parallel'' or ``parallel
tractability'' instead.

\begin{quote}
The abstract of the paper is much too long.  An abstract is supposed to be
one not-long paragraph.  Instead it is here two long paragraphs.  The
abstract needs to be cut down to a suitable size before any publication.
\end{quote}

Indeed, the abstract has been shortened significantly now.


\section{Rebuttal to Review 2}

\begin{quote}
First, in the discussion under Lemma 1, it says $|P^*|$ is polynomial in the size of $P$, which is imprecise, as $|P^*|$ is polynomial in the size of $I$ but not necessarily in the size of $R$. This is not a big issue if one considers NC for data complexity. What is less straightforward and more of concern is the statement that Step~2 costs only one time unit. As Step~2 involves checking the applicability of $B_1, \ldots, B_n \to H$ in $G$ and updating $G$, it depends on the size of $G$ and the size of $G$ is not a constant w.r.t.\ that of $I$. Footnote~4 suggests checking applicability of the rule can be done in constant time via an index. What about updating $G$ and maintaining the index?
\end{quote}

We rewrote the paragraph under Lemma~1 to analyze the parallel complexity of Algorithm~$\mathsf{A}_{\text{bsc}}$.
We made it clear that $|P^*|$ is polynomial in the size of $I$ for a class of datalog programs.\\

The statement that Step~2 costs only one time unit is based on the context of one processor. Since one processor
is allocated one rule instance (see the first paragraph of Algorithm~$\mathsf{A}_{\text{bsc}}$),
the procedure can be completed in constant time units. \\

The constant time of checking applicability by a processor is based on an assumption:
for any datalog program $P=\langle R, \textbf{I}\rangle$,
any substitution of some atom and any rule instance in $P^*$ can be mapped to a unique memory
location; further, a one-to-one relation can be established between processors and rule instances.
Under this assumption, a processor can check the applicability of its corresponding
rule instance and access the state of an atom occurring in this rule instance in constant time.
This assumption also applies to analyzing the computation of transitive closures \cite{Allender07} and
the problem of boolean matrix multiplication \cite{Raymond95}. Without this assumption, the program
costs at least linear time to load the inputs. Thus, the requirements of the \texttt{NC} algorithms
cannot be satisfied.\\

We provide a detailed description of the above assumption before Algorithm~$\mathsf{A}_{\text{bsc}}$.


\begin{quote}
The computation of the rch relation $S_{\text{rch}}$ ($\dag$) looks problematic, as it requires ``H has been added to G''. If it were the case, then in Example~4 the $S_{\text{rch}}$ would always be empty, as none of the rule heads $H$ would be added to $G$. Again, the complexity analysis of $A_{\text{opt}}$ is unclear and possibly flawed. Step~2 uses an NC algorithm to compute the transitive closure of $S_{\text{rch}}$, which runs in poly-logarithmic time w.r.t.\ the size of $S_{\text{rch}}$ not necessarily so w.r.t.\ the size of $I$. What is the size of $S_{\text{rch}}$ w.r.t.\ that of $I$? Similarly as above, it involves updating $G$, which depends on the sizes of both $G$ and $S_{\text{rch}}$. Would it be still in poly-logarithmic time?
\end{quote}

The requirement ``H has been added to G'' in the description of computing the rch relation $S_{\text{rch}}$ ($\dag$) is indeed
incorrect. It should be ``H has \textbf{not} been added to G''. We fixed this error.\\

We rewrote the paragraph following Lemma~2 to give a more detailed
analysis of the parallel complexity of Algorithm~$\mathsf{A}_{\text{opt}}$.
In Step~2 of Algorithm~$\mathsf{A}_{\text{opt}}$, an \texttt{NC} algorithm is used to compute
the transitive closure of $S_{\text{rch}}$.
It can be checked that the scale of $S_{\text{rch}}$ is polynomial in the size of inputs.
The total computing time is still in poly-logarithmic time (see the detailed analysis in the revision).


\begin{quote}
Finally, the algorithms assume each processor handling a rule instantiation. I wonder how it is implemented in ParallelDHL.
\end{quote}

We added a discussion of how each processor handles rule instantiations at the beginning of Section~6.2.



% We extended Section~6.1 to further discuss in what cases
% the two ontologies, YAGO and LUBM, are tractable in parallel.
% We also added the evaluation for these two kinds of ontologies in Section~6.2.\\

% The eight ontologies were selected since they satisfy the properties for parallel tractability.
% We used them to examine the effects of parallel tractability.
% In the revision, we did not continue to use the eight ontologies.
% We modified the LUBM ontology to generate ontologies of different graph depths.
% We gave the description of how these ontologies
% were constructed.\\

% In the new experiments, the results were clearer for explaining the issues of graph depths
% and parallel tractability.



\bibliographystyle{elsarticle-num}
\bibliography{response}

\end{document}

%%% Local Variables:
%%% mode: latex
%%% TeX-master: t
%%% End:
