\section{Conclusions}
\label{sec:conclusion}

In this paper, we studied the problem of parallel tractability of
materialization of datalog rewritable ontologies.
To identify classes that are tractable in parallel,
we proposed two \texttt{NC} algorithms, Algorithm~$\mathsf{A}_{bsc}^\psi$
and Algorithm~$\mathsf{A}_{opt}^\psi$, that perform materialization on
datalog rewritable ontology languages.
Based on these algorithms, we identified the corresponding classes of
datalog program for which materialization is tractable in
parallel (\texttt{PTD} classes), i.e., in the complexity class \texttt{NC}.
We further studied two specific ontology languages, DL-Lite and DHL (including an extension).
We showed that any ontology expressed in DL-Lite$_{\text{core}}$ or
DL-Lite$_{\mathcal{R}}$ is tractable in parallel.
For DHL and DHL($\circ$), we proposed two restrictions such that
materialization is tractable in parallel.

We verified the usefulness of our theoretical results in two ways. On the one hand,
we analyzed different kinds of datasets,
including well-known benchmarks, real-world ontologies and the famous YAGO dataset.
Our analysis showed that many real ontologies belong to
the class $\mathcal{D}_{\textit{\text{dhl}}}$ or $\mathcal{D}_{\textit{\text{dhl}}(\circ)}$, for which
materialization is tractable in parallel. The
developers and users can also refer to $\mathcal{D}_{\textit{\text{dhl}}}$
and $\mathcal{D}_{\textit{\text{dhl}}(\circ)}$
to create large-scale ontologies for which parallel tractability
is theoretically guaranteed. On the other hand,
we used an optimization strategy based on SWD paths
to give a practical algorithm variant Algorithm~$\mathsf{A}_{prc}$, which can
also be restricted to an \texttt{NC} version and can be implemented in practice.
We implemented a system based on Algorithm~$\mathsf{A}_{prc}$.
The experimental results showed that the optimizations proposed in this paper result
in a better performance on ontologies with parallel tractability compared
to the state-of-the-art reasoner RDFox.

In future work, we plan to extend our work in two lines.
One line is a further study of parallel tractability of
other expressive ontology languages, in addition to
datalog rewritable ones. Since different expressive OWL languages
have different syntaxes and higher reasoning complexities than \texttt{PTime}-complete, we need to
explore more restrictions that are practical and make materialization
tractable in parallel. Another line of work is to
study how to further apply the
results in practice.
One idea is to apply the technique of SWD paths to enhance other
reasoning-based tasks such as ontology classification, ontology debugging and query answering.




%%% Local Variables:
%%% mode: latex
%%% TeX-master: "parallel-tractability-J"
%%% End:
