\section{Conclusions}
\label{sec:conclusion}

In this paper, we studied the problem of parallel tractability of
materialization on datalog rewritable ontologies.
To identify the parallel tractability classes,
we proposed two \texttt{NC} algorithms, Algorithm~$\mathsf{A}_{\text{bsc}}^\psi$
and Algorithm~$\mathsf{A}_{\text{opt}}^\psi$, that perform materialization on
datalog rewritable ontology languages.
Based on these algorithms, we identified the corresponding
parallel tractability classes such that materialization
on the datalog programs in these classes is in \texttt{NC} complexity.
We further studied two specific ontology languages, DL-Lite and DHL (including one of its extension).
We showed that any ontology expressed in DL-Lite$_{core}$ or DL-Lite$_{\mathcal{R}}$ is tractable in parallel.
For DHL and DHL($\circ$), we proposed two restrictions such that materialization is tractable in parallel.

We analyzed the benchmark LUBM and the real-world dataset YAGO
and gave the cases where these ontologies belong to parallel tractability classes.
On the other hand, we used an optimization strategy based on SWD paths
to give a practical algorithm variant Algorithm~$\mathsf{A}_{\text{prc}}$, which can
also be restricted to an \texttt{NC} version.
We implemented a system based on Algorithm~$\mathsf{A}_{\text{prc}}$ and compared it to
the state-of-the-art reasoner RDFox.
The experimental results showed that the optimizations proposed in this paper result in a
better performance on ontologies that are
tractable in parallel compared to RDFox.

In future work, we plan to extend our work in two lines.
One line is a further study of parallel tractability of
other expressive ontology languages, in addition to
datalog rewritable ontology languages. Since different expressive OWL languages
have different syntaxes and higher reasoning complexities than \texttt{PTime}-complete, we need to
explore more restrictions that are practical and make materialization
tractable in parallel. Another line of work is to
study how to further apply the
results in practice.
One idea is to apply the technique of SWD paths to enhance other
reasoning-based tasks such as ontology classification, ontology
debugging and query answering.




%%% Local Variables:
%%% mode: latex
%%% TeX-master: "parallel-tractability-J"
%%% End:
