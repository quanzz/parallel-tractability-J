\section{Conclusions}
\label{sec:conclusion}

In this paper, we studied the problem of parallel tractability of
materialization on the datalog rewritable ontologies.
To identify the parallelly tractable classes,
we proposed two NC algorithms, Algorithm~$\mathsf{A}_{bsc}^\psi$
and Algorithm~$\mathsf{A}_{opt}^\psi$, that perform materialization on
datalog rewritable ontology languages.
Based on these algorithms, we identified the corresponding
parallelly tractable datalog program (\texttt{PTD}) classes such that materialization
on the datalog programs in these classes is in the complexity class NC.
We further studied two specific ontology languages, DL-Lite and DHL (including one of its extension).
We showed that any ontology expressed in DL-Lite$_{core}$ or DL-Lite$_{\mathcal{R}}$ is parallelly tractable.
For DHL and DHL($\circ$), we proposed two restrictions such that materialization is parallelly tractable.

We verified the usefulness of our theoretical techniques in two ways. On the one hand,
we analyzed different kinds of datasets,
including well-known benchmarks, real-world ontologies and a famous dataset YAGO.
Our analysis showed that many real ontologies belong to
the parallelly tractable class $\mathcal{D}_{\textit{\text{dhl}}}$
or $\mathcal{D}_{\textit{\text{dhl}}(\circ)}$. The
developers and users can also refer to $\mathcal{D}_{\textit{\text{dhl}}}$
and $\mathcal{D}_{\textit{\text{dhl}}(\circ)}$
to create large-scale ontologies for which parallel tractability
is theoretically guaranteed. On the other hand,
we used an optimization strategy based on SWD paths
to give a practical algorithm variant Algorithm~$\mathsf{A}_{prc}$, which can
also be restricted to an NC version, and can be implemented for practice.
We implemented a system based on Algorithm~$\mathsf{A}_{prc}$ and compared it to
the state-of-the-art reasoner RDFox.
The experimental results showed that the optimization proposed in this paper results
in a better performance on parallelly tractable ontologies compared with RDFox.

In the future work, we will extend our work in two lines.
One line is a further study of the parallel tractability of
other expressive ontology languages, in addition to
datalog rewritable ones. Since different expressive OWL languages
have different syntaxes and higher reasoning complexities than \texttt{PTime}-complete, we need to
explore more restrictions that are practical and make materialization
parallelly tractable. Another line of the future work is to
study how to further apply the
results in practice.
One idea is to apply the technique of SWD paths to enhance other reasoning-based tasks,
like ontology classification, ontology debugging and query answering.




%%% Local Variables:
%%% mode: latex
%%% TeX-master: "parallel-tractability-J"
%%% End:
