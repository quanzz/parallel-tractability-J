\section{Related Work}
\label{sec:related}

The parallel reasoner RDFox \cite{MotikNPHO14} is used to evaluate our implementation. RDFox is a state-of-the-art system that handles reasoning on datalog rewritable ontology languages. Algorithm~$\mathsf{A}_{bsc}$ proposed in Section~\ref{sec:ptclass}, is similar to the main algorithm of RDFox (see \cite{MotikNPHO14}, Sections~3 and 4). The main difference is that a group of rule instantiations is handled by one processor (namely a thread) in RDFox, while in Algorithm~$\mathsf{A}_{bsc}$, each rule instantiation is assigned to a unique processor. Thus, in theory, the materialization of the datalog program in Example~\ref{exp:mg} is serial in RDFox {\color{red}The example has two rules. Why can they not be parallelized in RDFox?}. ItThe experiments also show that parallelism does not lead to a remarkable improvement for RDFox for the ontology series not in $\mathcal{D}_{\textit{\text{dhl}}(\circ)}$. We use Algorithm~$\mathsf{A}_{opt}$ to show that the ontology in Example~\ref{exp:mg} is also tractable in parallel. Our implementation ParallelDHL, which is based on Algorithm~$\mathsf{A}_{prc}$, also shows a better performance than RDFox when handling the test ontologies that belong to $\mathcal{D}_{\textit{\text{dhl}}(\circ)}$.

There is work that studies parallel reasoning in RDFS and OWL. The current methods mainly focus on optimizing reasoning algorithms from different aspects. \citet{GoodmanJMAAH11} propose a new kind of encoding method for RDF triples to achieve a high performance. This method can significantly yield a throughput improvement and optimize parallel RDF reasoning and query answering. \citet{PetersSZ15} use the RETE algorithm to accelerate rule matching for RDFS reasoning. \citet{SubercazeGCL16} propose a more efficient storage technique and optimize the join operations in parallel reasoning. The issue of balance distribution of parallel tasks is also studied \cite{SomaP08,WeaverH09}. Two approaches are explored, i.e., \emph{rule partitioning} (allocating parallel tasks to different processors based on rules) and \emph{data partitioning} (allocating parallel tasks based on data). The evaluation results indicate that the efficiency of balance distribution varies with respect to different datasets. On the other hand, parallel reasoning is also implemented for OWL fragments, e.g., OWL~RL \cite{KolovskiWE10}, OWL~EL \cite{KazakovKS14}, OWL~QL \cite{LemboSS13}, and even highly expressive languages \cite{SteigmillerLG14,LiebigM07,SchlichtS08,WuH12}. In current work, several techniques are proposed to adapt parallel computation to OWL reasoning tasks. A kind of graph-based method is discussed by \citet{LemboSS13} to enhance OWL~QL classification.  Several pruning techniques are proposed to optimize the Tableaux algorithm \cite{LiebigM07,SchlichtS08,WuH12}. The \emph{lock-free technique} is applied by \citet{KazakovKS14} and \citet{SteigmillerLG14}.

Another line of optimizing parallel reasoning is to utilize high-performance computing platforms. For in-memory platforms, different supercomputers such as Cray XMT or Yahoo S4 have early on been used in parallel RDF reasoning \cite{Hoeksema2011,GoodmanJMAAH11}. \citet{HeinoP12} report on their work on RDFS reasoning based on massively parallel GPU hardware. Distributed parallel platforms such as MapReduce and Peer-to-Peer networks are also used for RDFS reasoning. The representative systems are WebPIE \cite{UrbaniKMHB12}, Marvin \cite{oren2009marvin} and SAOR \cite{HoganHP09}. Different techniques are discussed in this work to tackle the special problems in distributed computing. However, to study the issue of parallel tractability on distributed platforms,  more issues have to be discussed, e.g., \emph{network structures} and \emph{communications}. This is not the focus in this paper.

Different from the above work, the purpose of this paper is to study the issue of the parallel tractability of materialization from the perspective of data. The results given in this paper guarantee the parallel tractability theoretically, regardless of what optimization techniques and platforms as discussed above are used.




%%% Local Variables:
%%% mode: latex
%%% TeX-master: "parallel-tractability-J"
%%% End:
