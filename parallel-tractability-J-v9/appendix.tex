\section*{Appendix}

\appendix

\section{Proof of Lemma~\ref{lemma:a1}}

\textbf{Lemma~\ref{lemma:a1}}
\emph{Given a datalog program $P=\langle R, \textbf{I}\rangle$, we have
(1) $\mathsf{A}_{bsc}$ halts and returns a materialization graph $\mathcal{G}$ of $P$;
(2) $\mathcal{G}$ has the the minimum depth among all the materialization graphs of $P$.}\\

\noindent\emph{Proof}:
(1) First, whenever a processor $p$ adds a new node $v$ to $\mathcal{G}$, it has to first
check whether $v$ has already been in $\mathcal{G}$ and does nothing if $v$ is in $\mathcal{G}$.
Thus $\mathcal{G}$ turns out to be an acyclic graph.
Second, to show $\mathcal{G}$ is a
complete materialization graph, we perform an induction on $T_R^{\omega}(\textbf{I})$.
Specifically, all the facts in $T_R^{0}(\textbf{I})$ have to be in $\mathcal{G}$ by \ref{alg1:addFacts}
of $\mathsf{A}_{bsc}$.
For $i>0$, suppose that the ground atoms in $T_R^{i}(\textbf{I})$ are in $\mathcal{G}$.
It can be checked that the atoms in $T_R^{i+1}(\textbf{I})$ have to be
added to $\mathcal{G}$, since whenever a new node is derived from $T_R^{i}(\textbf{I})$,
there has to be a processor that would add it to $\mathcal{G}$.

(2) The \emph{stage} (see the related contents in Section~2.3) of $P$ is the lower bound of the depth
of all materialization graphs. One can further check that, for the materialization graph $\mathcal{G}$
constructed by $\mathsf{A}_{bsc}$, its depth equals to the stage based on the induction above.
Thus, $\mathcal{G}$ has the minimal depth among all the materialization graphs of $P$. \hfill$\Box$

\section{Proof of Theorem~\ref{theorem:a1}}

\textbf{Theorem~\ref{theorem:a1}}
\emph{For any datalog program $P$, $P\in\mathcal{D}_{\mathsf{A}_{bsc}^{\psi}}$ iff
$P$ has a materialization graph whose depth is upper-bounded by $\psi(|P|)$.}\\

\noindent\emph{Proof:} We first prove that the number of iterations of \ref{alg1:updateG}
is actually the depth of the constructed materialization graph.
We define the depths of nodes iteratively as follows:
for each explicit node $v$, \texttt{depth}($v$)=0;
for each implicit node $v'$ whose parents are $v_1,...,v_i$, \texttt{depth}($v'$)=$\max$$\{$ \texttt{depth}($v_1$), ..., \texttt{depth}($v_i$) $\}+1$.
By performing an induction on the number of iterations of \ref{alg1:updateG},
one can check that an implicit node $v'$ has to be added to $\mathcal{G}$ in the $n^{th}$ iteration
where $n$=\texttt{depth}($v'$).
Further, \texttt{depth}($\mathcal{G}$)=$\max$$\{$\texttt{depth}($v_i$)$|v_i$ is in $\mathcal{G}\}$.
We then have that the number of iterations of \ref{alg1:updateG} is \texttt{depth}($\mathcal{G}$).

($\Rightarrow$) $P\in\mathcal{D}_{\mathsf{A}_{bsc}^{\psi}}$ means that $\mathsf{A}_{bsc}^{\psi}$ can return
a materialization graph $\mathcal{G}$ of $P$.
Recall that the number of iterations of \ref{alg1:updateG} is bounded by $\psi(|P|)$.
Thus \texttt{depth}($\mathcal{G}$) is also bounded by $\psi(|P|)$.

($\Leftarrow$) Suppose $P$ has a materialization graph $\mathcal{G}$ whose depth
is upper-bounded by $\psi(|P|)$.
If $\mathcal{G}$ has the minimal depth among other materialization graphs of $P$,
the number of iterations of \ref{alg1:updateG} is also \texttt{depth}($\mathcal{G}$) (Lemma~\ref{lemma:a1})
and, thus, upper-bounded by $\psi(|P|)$.
If $\mathsf{A}_{bsc}$ dose not return $\mathcal{G}$, then the returned graph $\mathcal{G}'$
should have a smaller depth compared with $\mathcal{G}$.
In this case, this conclusion still holds.\hfill$\Box$

\section{Proof of Lemma~\ref{lemma:a3}}

\textbf{Lemma~\ref{lemma:a3}}
\emph{Given a datalog program $P=\langle R, \textbf{I}\rangle$,
$\mathsf{A}_{opt}$ halts and the output $\mathcal{G}$ is a materialization graph of $P$.}\\

\noindent\emph{Proof}: this lemma can be proved by two stages:
(1) the graph $\mathcal{G}$ returned by $\mathsf{A}_{opt}$ is a materialization graph;
(2) $\mathcal{G}$ is a complete materialization graph.
We first show that (1) holds by an induction on the iterations of \ref{alg3:updateG} of $\mathsf{A}_{opt}$.

\emph{Base case}. Initially, $\mathcal{G}$ only contains all the explicit nodes.
In this case, $\mathcal{G}$ is obviously a materialization graph.

\emph{Inductive case}.
According to the induction hypothesis, the partial graph constructed after
the $i^{th}$ ($i>1$) iteration is a materialization graph, denoted by $\mathcal{G}^{i}$.
We have to show that the partial graph constructed after the $i+1^{th}$ iteration is
also a materialization graph, denoted by $\mathcal{G}^{i+1}$.
The partial graph $\mathcal{G}^{i+1}$ is updated in $\mathsf{A}_{opt}$ by performing the step $(\textbf{\romannumeral3})$
of Algorithm~$\mathsf{Opt}$.
Suppose that \texttt{rch}$(B_k,H)\in S_{\textit{\!\tiny rch}}$ is checked.
It corresponds to the rule instantiation $B_1,..,B_k,..,B_n\rightarrow H$.
Algorithm~$\mathsf{Opt}$ next checks that there exists a node $B'$ such that
\texttt{rch}$(B',B_k)\in S^*_{\!\tiny rch}$
and $B'$ is in $\mathcal{G}$. This means that $B_k$ and $H$ are derivable.
The algorithm then adds new nodes $B_k$ and $H$ to $\mathcal{G}^{i}$ in three cases.
(1) if $H$ is not in $\mathcal{G}^{i}$ while $B_k$ is in $\mathcal{G}^{i}$,
then $H$ is added to $\mathcal{G}^{i+1}$, and the edges $e(B_1, H),...,e(B_n, H)$ are created.
(2) when neither of $H$ and $B_k$ is in $\mathcal{G}^{i}$,
then $H$ and $B_k$ are added to $\mathcal{G}^{i+1}$, and the edges $e(B_1, H),...,e(B_n, H)$ are also created.
In the above two cases, $H$ is a new node. Thus,
$\mathcal{G}^{i+1}$ is acyclic.
(3) if $H$ is in $\mathcal{G}^{i}$ and $H$ has no parent,
$B_k$ is added to $\mathcal{G}^{i+1}$, and the edges $e(B_1, H),...,e(B_n, H)$ are also created.
For the case that $H$ is in $\mathcal{G}^{i}$ and $H$ has parents,
the algorithm does nothing.
Thus, $\mathcal{G}^{i+1}$ is acyclic and satisfies the definition of materialization graph.

To show that (2) holds, we use the same method in the proof for Lemma~\ref{lemma:a1}.
We want to show that all the ground atoms in $T_R^{i+1}(\textbf{I})$ have to be added to $\mathcal{G}$,
with the induction hypothesis that the ground atoms in $T_R^{i}(\textbf{I})$ are in $\mathcal{G}$.
Suppose the ground atoms in $T_R^{i}(\textbf{I})$ have been added to $\mathcal{G}$ by
performing $\mathsf{A}_{opt}$.
For each atom $\alpha$ in $T_R^{i+1}(\textbf{I})$, $\alpha$ actually has a special SWD path
of the length 1. This is because that all parents of $\alpha$ have been in $\mathcal{G}$.
According to the optimization strategy, $\alpha$ has to be added to $\mathcal{G}$ by applying \ref{alg3:updateG}
of $\mathsf{A}_{opt}$.\hfill$\Box$


\section{Proof of Lemma~\ref{lemma:dllite}}

\textbf{Lemma~\ref{lemma:dllite}}
\emph{Given a DL-Lite ontology $\mathcal{O}$, for any materialization graph $\mathcal{G}$ of $\mathcal{O}$,
each atom of the form $A(x)$ ($A\neq\bot$) or $R(x,y)$ in $\mathcal{G}$ has an SWD path.}\\

\noindent\emph{Proof}: this lemma can be proved by an induction on applications of
the datalog rules corresponding to DL-Lite axioms.

\emph{Base case}. For each explicit node $v$ of the form $A(x)$ ($A\neq\bot$) or $R(x,y)$,
$v$ has a special SWD path with $v$ as the unique node. This conclusion holds
for any materialization graph.

\emph{Inductive cases}. For each datalog rule of the form `$B_1,...,B_n\rightarrow H$'
that is rewritten from some DL-Lite axiom, we have the induction hypothesis
that $B_1,...,B_n$ have SWD paths.
We are left to prove that $H$ has an SWD path as well
in all materialization graphs of $\mathcal{O}$.

If $H$ is in the form of $A(x)$ ($A\neq\bot$), it may be derived by
applying (T1) and (T3). Thus, we conduct the induction by
distinguish these two cases as follows.

\begin{enumerate}[leftmargin=12ex,label=Case~1.\arabic*]
\item $B\sqsubseteq A$. According to the induction hypothesis,
node $B(x)$ has an SWD path, denoted by $(v_1,...,v_n,B(x))$.
We have that, in some materialization graph, node $A(x)$
has an SWD path of the form $(v_1,...,v_n,B(x),A(x))$.

\item $\exists R\sqsubseteq A$. Node $R(x,y)$ has an SWD path,
denoted by $(v_1,...,v_n,R(x,y))$,
according to the induction hypothesis.
It is obvious that node $A(x)$
has an SWD path of the form $(v_1,...,v_n,B(x),R(x,y))$
in some materialization graph.
\end{enumerate}

Since node $A(x)$ can only be derived in either Case~1.1
or Case~1.2, we have that
node $A(x)$ has an SWD path in all materialization graphs of $\mathcal{O}$.

It is similar to prove the case where $H$ is in the form of $R(x,y)$.
Since node $R(x,y)$ may be derived by
applying (R1) and (R2), we discuss these two cases.

\begin{enumerate}[leftmargin=12ex,label=Case~2.\arabic*]
\item $S\sqsubseteq R$. According to the induction hypothesis,
node $S(x,y)$ has an SWD path, denoted by $(v_1,...,v_n,S(x,y))$.
Obviously, node $R(x,y)$
has an SWD path of the form $(v_1,...,v_n,S(x,y),R(x,y))$
in some materialization graph.

\item $S\sqsubseteq R^-$. This case is similar to
the above case.
\end{enumerate}

In both of Case~2.1 and Case~2.2, node $R(x,y)$ has an SWD path. Thus,
node $R(x,y)$ has an SWD path in all materialization graphs of $\mathcal{O}$.\hfill$\Box$

\section{Proof of Theorem~\ref{theorem:dl-lite}}

\textbf{Theorem~\ref{theorem:dl-lite}}
\emph{There exists a poly-logarithmically bounded function $\psi$ s.t.
$\mathcal{D}_{\textit{\text{dl-lite}}}\subseteq\mathcal{D}_{\mathsf{A}_{opt}^{\psi}}$.}\\

\noindent\emph{Proof}: this theorem can be easily proved based on Lemma~\ref{lemma:dllite}.
Specifically, for any DL-lite ontology $\mathcal{O}$,
there always exists a poly-logarithmically bounded function $\psi$ such that
$\mathsf{A}_{opt}^{\psi}$ can handle the materialization of $\mathcal{O}$,
since in any materialization graph, each node has an SWD path.
More precisely, we can set that $\psi=2$.\hfill$\Box$


\section{Proof of Theorem~\ref{theorem:dhl}}

\textbf{Theorem~\ref{theorem:dhl}}
\emph{There exists a poly-logarithmically bounded function $\psi$ s.t.
$\mathcal{D}_{\textit{\text{dhl}}}\subseteq\mathcal{D}_{\mathsf{A}_{opt}^{\psi}}$.}\\

\noindent\emph{Proof}: observe that, for a datalog program that is transformed from a DHL ontology,
a rule with a binary atom as its head can only be either of the form (R1), (R2) or (R3).
This also indicates that, the materialization of a DHL
ontology can be separated into two stages: in the first stage (Stage~1), all the rules
of the forms (R1-R3) are exhaustively applied; the consequences of the first stage
also serve as the facts in the second stage (Stage~2). The rules of the forms (T1-T3)
are then applied in Stage~2.
It is obvious that, if both of Stage~1 and Stage~2 can be handled
by performing $\mathcal{A}_{opt}^\psi$ for some poly-logarithmical function $\psi$, then
the whole materialization can be handled by $\mathcal{A}_{opt}^\psi$.
In what follows, we investigate the above two stages respectively, and show that,
for any datalog program in $\mathcal{D}_{\textit{\text{dhl}}}$,
such a poly-logarithmical function $\psi$ exists.

In Stage~1, the rules of the forms (R1-R3)
are applied to add new nodes to the constructed materialization graph.
We can observe that, rule (R3) is used for computation of transitive roles.
As mentioned before, there exists an NC algorithm for transitivity computation.
Inspired by this, we can prove that \emph{$\mathcal{A}_{opt}^\psi$
handles Stage~1 for some poly-logarithmical function $\psi$}.
The proof of this result can be shown by separately considering non-transitive roles and
the roles that are transitive or influenced by transitive roles.
Specifically, we say that role $R$ is \emph{transitively influenced} (TI)
if (1) $R\circ R\sqsubseteq R\in\mathcal{R}$; or
(2) there exists a TI role $R'$ such that $R'\sqsubseteq_*R$ or $R'\sqsubseteq_*R^-$.
We say that a role is an NTI (non-transitively influenced) role if it is not a TI role.
We further define a set $\delta_R$ for each role $R\in\textbf{R}$ as follows:

\begin{definition}\label{def:deltaRdhl}
For each $R\in\textbf{R}$, let $\delta_R$ be the set of all assertions as follows:
\begin{enumerate}[leftmargin=4ex,label=\arabic*.]
\item for each $R(a,b)\in\mathcal{A}$;
\item $R(a,b)$, for each $R'(a,b)\in\mathcal{A}$ and $R'\sqsubseteq_{*}R$;
\item $R(a,b)$, for each $R'(b,a)\in\mathcal{A}$ and $R'\sqsubseteq_{*}R^-$.
\end{enumerate}
\end{definition}

Let $\delta^*_R$ be the transitive closure of $\delta_R$ where $R$ is a transitive role.
We then have the following lemma.

\begin{lemma}\label{lemma:ti}
$P\models R(a,b)$ implies: (1) if $R$ is an NTI role, $R(a,b)\in\delta_R$;
(2) if $R$ is a transitive role, $R(a,b)\in\delta^*_{R}$;
(3) if $R$ is a TI role, then $R(a,b)\in\delta_R$, or there exists a transitive
role $R'$ such that $R'\sqsubseteq_* R$ and $R'(a,b)\in\delta^*_{R'}$.
\end{lemma}

Note that, for all roles $R$, $\delta_R$ can be computed by only applying (R1) and (R2).
In this sense, Lemma~\ref{lemma:ti} also indicates that:
(1) for each implicit node $R(a,b)$ where $R$ is an NTI role,
it can be added to a materialization graph by only applying (R1) and (R2);
(2) for each transitive role $R$, all implicit nodes of the form $R(a,b)$ are in the transitive closure $\delta^*_{R}$,
which can be computed by an NC algorithm on $\delta_{R}$;
(3) for each role $R$ that is a TI role but not a transitive role,
one can further perform (R1) and (R2) iteratively based on all transitive closures $\delta^*_{R'}$
where $R'$ is a transitive role.
Since all nodes generated by only applying (R1) and (R2) have SWD paths,
the computations of (1) and (3) can be handled by $\mathcal{A}_{opt}^\psi$.
Further, transitive computation in part (2) can also be handled by $\mathcal{A}_{opt}^\psi$.
Thus, there exists a poly-logarithmical function $\psi$
such that $\mathcal{A}_{opt}^\psi$ handles Stage~1.

The results of Stage~1 serve as the facts of Stage~2. In other words, all binary atoms
are explicit nodes in Stage~2. This also means that the rules of the forms (T1) and (T3)
generate nodes with SWD paths. Due to the simple-concept restriction, for each rule
instantiation $A_1(a),A_2(a)\rightarrow B(a)$, one of $A_1(a)$ and $A_2(a)$ always has an SWD path.
Thus, Stage~2 can be handled by $\mathcal{A}_{opt}$ in at most two iterations.
Based on the above analysis, this theorem holds.

We are now left to prove Lemma~\ref{lemma:ti}. We conduct the proof by an induction on
the derivation of $P=\langle R,\textbf{I}\rangle$. We distinguish inductive cases by different rules (R1-R3)
that are possibly applied to derive $R(a,b)$.

\emph{Basic case}. If $R(a,b)\in\textbf{I}$, $R(a,b)\in\delta_{R}$. In this case,
regardless of that $R$ is a TI or an NTI role, all of (1), (2) and (3) hold.

\emph{Inductive case}. We first study the case where $R$ is an NTI role.

\begin{enumerate}[leftmargin=12ex,label=Case~1.\arabic*, ref=Case~1.\arabic*]
\item $R(a,b)$ is derived by applying the rule (R1), w.l.o.g., $R'(a,b)\rightarrow R(a,b)$,
    which also means that $R'\sqsubseteq R\in\mathcal{R}$. Since $R$ is an NTI role, $R'$ has to
    be an NTI role according to Definition~\ref{def:deltaRdhl}. According to the induction hypothesis,
    $R'(a,b)\in\delta_{R'}$. (Case~1.1.1) If $R'(a,b)\in\mathcal{A}$, $R(a,b)\in\delta_{R}$
    holds according to Definition~\ref{def:deltaRdhl}; (Case~1.1.2) If there exists some
    role $R''$ such that $R''(a,b)\in\mathcal{A}$ and $R''\sqsubseteq R'$, we also have
    $R(a,b)\in\delta_{R}$ since $R''\sqsubseteq_* R$ holds; (Case~1.1.3) If there exists some
    role $R''$ such that $R''(b,a)\in\mathcal{A}$ and $R''\sqsubseteq R'^-$, this case is
    similar to (Case~1.1.2).\label{dhl:NTIcaseR1}

\item $R(a,b)$ is derived by applying the rule (R2), w.l.o.g., $R'(a,b)\rightarrow R(b,a)$,
    which also means that $R'\sqsubseteq R^-\in\mathcal{R}$. This case is similar
    to \ref{dhl:NTIcaseR1}.\label{dhl:NTIcaseR2}

\item $R(a,b)$ is derived by applying the rule (R3). This case is impossible, since $R$ is
    an NTI role.
\end{enumerate}

We next study the case where $R$ is a transitive role.

\begin{enumerate}[leftmargin=12ex,label=Case~2.\arabic*, ref=Case~2.\arabic*]
\item $R(a,b)$ is derived by applying the rule (R1), w.l.o.g., $R'(a,b)\rightarrow R(a,b)$,
    which also means that $R'\sqsubseteq R\in\mathcal{R}$.\label{dhl:TcaseR1}
    \begin{enumerate}[leftmargin=8ex,label=Case~2.1.\arabic*]
    \item $R'$ is an NTI role. By the induction hypothesis, $R'(a,b)\in\delta_{R'}$.
        Similar to (Case~1.1.1), $R(a,b)\in\delta_{R}$ holds.
        Obviously, $R(a,b)\in\delta_{R}^*$ also holds.

    \item $R'$ is a TI role. By the induction hypothesis, if $R'(a,b)\in\delta_{R'}$,
        then $R(a,b)\in\delta_R$ also holds. On the other hand, if there exists a transitive role
        $R''$ (if $R'$ is a transitive role, then $R''\equiv R'$) such that
        $R''\sqsubseteq_* R'$ and $R''(a,b)\in\delta_{R''}^*$. Since $\delta_{R''}^*$ is
        the transitive closure of $\delta_{R''}$, we then have that there must
        exist such atoms ($R''(a,c_1),R''(c_1,c_2),...,R''(c_n,b)$) in $\delta_{R''}$.
        Further, due to $R''\sqsubseteq_* R$ and Definition~\ref{def:deltaRdhl},
        we have $R(a,c_1),R(c_1,c_2),...,$ $R(c_n,b)\in\delta_{R}$. Thus, $R(a,b)\in\delta_{R}^*$
        also holds.
    \end{enumerate}

\item $R(a,b)$ is derived by applying the rule (R2), w.l.o.g., $R'(b,a)\rightarrow R(a,b)$,
    which also means $R'\sqsubseteq R^-\in\mathcal{R}$.\label{dhl:TcaseR2}
    \begin{enumerate}[leftmargin=8ex,label=Case~2.2.\arabic*]
    \item $R'$ is an NTI role. By induction hypothesis, $R'(b,a)\in\delta_{R'}$.
        Similar to \ref{dhl:NTIcaseR2}, $R(a,b)\in\delta_{R}$ holds.
        Obviously, $R(a,b)\in\delta_{R}^*$ also holds.

    \item $R'$ is a TI role. By the induction hypothesis, if $R'(b,a)\in\delta_{R'}$,
        then $R(a,b)\in\delta_R$ also holds. On the other hand, if there exists a transitive role
        $R''$ (if $R'$ is a transitive role, then $R''\equiv R'$) such that
        $R''\sqsubseteq_* R'$ and $R''(b,a)\in\delta_{R''}^*$. Since $\delta_{R''}^*$ is
        the transitive closure of $\delta_{R''}$, we then have that there must
        exist such atoms ($R''(b,c_1),R''(c_1,c_2),...,R''(c_n,a)$) in $\delta_{R''}$.
        Further, due to $R''\sqsubseteq_* R^-$ and Definition~\ref{def:deltaRdhl},
        we have $R(a,c_n),R(c_n,c_{n-1}),...,$ $R(c_1,b)\in\delta_{R}$. Thus, $R(a,b)\in\delta_{R}^*$
        also holds.
    \end{enumerate}

\item $R(a,b)$ is derived by applying the rule (R3), w.l.o.g., $R(a,c),R(c,b)\rightarrow R(a,b)$,
    which also means that $R\circ R\sqsubseteq R\in\mathcal{R}$. By the induction hypothesis,
    $R(a,c),R(c,b)\in\delta_{R}^*$ holds. Obviously, $R(a,b)\in\delta_{R}^*$ also holds.
\end{enumerate}

We finally study the case where $R$ is a TI but not a transitive role.

\begin{enumerate}[leftmargin=12ex,label=Case~3.\arabic*, ref=Case~3.\arabic*]
\item $R(a,b)$ is derived by applying the rule (R1), w.l.o.g., $R'(a,b)\rightarrow R(a,b)$,
    which also means that $R'\sqsubseteq R\in\mathcal{R}$.\label{dhl:NTcaseR1}
    \begin{enumerate}[leftmargin=8ex,label=Case~3.1.\arabic*]
    \item $R'$ is an NTI role. By the induction hypothesis, $R'(a,b)\in\delta_{R'}$ holds.
        Similar to (Case~1.1.1), $R(a,b)\in\delta_{R}$ holds.
        Obviously, $R(a,b)\in\delta_{R}^*$ also holds.

    \item $R'$ is a TI role. By the induction hypothesis, if $R'(a,b)\in\delta_{R'}$,
        then $R(a,b)\in\delta_R$ also holds. On the other hand, if there exists a transitive role
        $R''$ (if $R'$ is a transitive role, then $R''\equiv R'$) such that
        $R''\sqsubseteq_* R'$ and $R''(a,b)\in\delta_{R''}^*$. Since $R''\sqsubseteq_* R$ and
        $R''$ is the transitive role, third consequence in this Lemma is satisfied.
    \end{enumerate}

\item $R(a,b)$ is derived by applying the rule (R2), w.l.o.g., $R'(a,b)\rightarrow R(b,a)$,
    which also means that $R'\sqsubseteq R^-\in\mathcal{R}$. This case is similar to \ref{dhl:NTcaseR1}.

\item $R(a,b)$ is derived by applying the rule (R3). This is impossible, since $R$ is not a transitive
    role.\hfill$\Box$
\end{enumerate}


\section{Proof of Theorem~\ref{theorem:dhlplus}}

\textbf{Theorem~\ref{theorem:dhlplus}}
\emph{There exists a poly-logarithmically bounded function $\psi$ s.t. $\mathcal{D}_{\textit{\text{dhl}}(\circ)}\subseteq\mathcal{D}_{\mathsf{A}_{opt}^{\psi}}$.}\\

\noindent\emph{Proof}: the proof idea of this theorem is similar to that of Theorem~\ref{theorem:dhl}.
That is, we separate the materialization of DHL$(\circ)$ ontologies into two stages: in Stage~1,
all the rules of the forms (R1-R4) are exhaustively applied; in Stage~2, the rules of the forms (T1-T2)
are then applied while the results of Stage~1 serve as facts. Stage~2 is as same as that of DHL.
Thus we only consider Stage~1 here.

Our target is to show that $\mathcal{A}_{opt}^\psi$ handles Stage~1. To this end, we also
distinguish all roles by whether they are transitively influenced. Since we have to consider
complex RIAs, we re-define TI and NTI roles as follows:
a role $R$ is \emph{transitively influenced} (TI)
if (1) $R\circ R\sqsubseteq R\in\mathcal{R}$; or
(2) there exists a TI role $R'$ such that $R'\sqsubseteq_*R$ or $R'\sqsubseteq_*R^-$;
or (3) there exist a TI role $R'$ and an axiom of either of the form
$R'\circ R''\sqsubseteq R$ or $R''\circ R'\sqsubseteq R$.
We say that a role is an NTI role if it is not transitively influenced.
The set $\delta_R$ for each role $R\in\textbf{R}$ is re-defined based on \emph{role sequence set}
that is defined as follows:

\begin{definition}\label{def:language}
Let $\mathcal{L}(R)$ be the set of role sequences with respect to $R$ as follows:
 \begin{enumerate}[leftmargin=4ex,label=\arabic*.]
\item $R'$, for each $R'\sqsubseteq_* R$;
\item $R'^-$, for each $R'\sqsubseteq_* R^-$;
\item $L_1L_2$, for each axiom of the form (except R3) $R_1\circ R_2\sqsubseteq R'\in\mathcal{R}$, $L_i\in\mathcal{L}(R_i)(1\leq i\leq 2)$ and $R'\in\mathcal{L}(R)$;
\item $L_2^-L_1^-$, for each axiom of the form (except R3) $R_1\circ R_2\sqsubseteq R'\in\mathcal{R}$, $L_i\in\mathcal{L}(R_i)(1\leq i\leq 2)$ and $R'^-\in\mathcal{L}(R)$.
\end{enumerate}
\end{definition}

In the above definition, for a role sequence $L=R_1R_2,...,R_n$, let $L^-=R_n^-,...,R_2^-R_1^-$.
We then give the following definition for $\delta_R$.

\begin{definition}\label{def:deltaRdhlplus}
For each $R\in\textbf{RN}$, let $\delta_R$ be the set of all assertions as follows:
\begin{enumerate}[leftmargin=4ex,label=\arabic*.]
\item for each $R(a,b)\in\mathcal{A}$;
\item $R(a,b)$, for each $R'(a,b)\in\mathcal{A}$ and $R'\sqsubseteq_{*}R$;
\item $R(a,b)$, for each $R'(b,a)\in\mathcal{A}$ and $R'\sqsubseteq_{*}R^-$;
\item $R(a,b)$, for each $R_0R_2,...,R_n\in\mathcal{L}(R)$, where $R_i(x_i,x_{i+1})\in\mathcal{A}$ and $x_0=a, x_{n+1}=b$ for $0\leq i\leq n$.
\end{enumerate}
\end{definition}

\begin{lemma}\label{lemma:tiplus}
$P\models R(a,b)$ implies: (1) if $R$ is an NTI role, $R(a,b)\in\delta_R$;
(2) if $R$ is a transitive role, then $R(a,b)\in\delta^*_R$;
(3) if $R$ is a TI but not transitive role, then there exists a role sequence $R_1R_2,...,R_n\in\mathcal{L}(R)$ such that
$R_i(x_i,x_{i+1})\in\delta_{R_i}$, or, $R_i(x_i,x_{i+1})\in\delta^*_{R_i}$ if $R_i$ is a transitive role,
where $1\leq i\leq n$ and $x_0=a, x_{n+1}=b$.
\end{lemma}

The set $\delta^*_R$ also denotes the transitive closure of $\delta_R$ where $R$ is a transitive role.
Note that, for all roles $R$, $\delta_R$ can be computed by applying (R1), (R2) and (R4).
The above lemma says:
(1) for each implicit node $R(a,b)$ where $R$ is an NTI role,
it can be added to a materialization graph by only applying (R1), (R2) and (R4);
(2) for transitive roles $R$, all implicit nodes are in the transitive closure $\delta^*_{R}$,
which can be computed by an NC algorithm on $\delta_{R}$;
(3) for each role $R$ that is a TI but not a transitive role,
one can further perform (R1), (R2) and (R4) iteratively based on all
transitive closures $\delta^*_{R'}$ where $R'$ is a transitive role.
All nodes generated by applying (R1) and (R2) have SWD paths in the first iteration; at
least one role in the left hand side of (R4) are restricted to be a simple role.
Thus, the computations of (1) and (3) can be handled by $\mathcal{A}_{opt}^\psi$.
Further, transitive computation in part (2) can also be handled by $\mathcal{A}_{opt}^\psi$.
In summary, there exists a poly-logarithmical function $\psi$
such that $\mathcal{A}_{opt}^\psi$ handles Stage~1.

We now prove Lemma~\ref{lemma:tiplus}.
We conduct the proof by an induction on
the derivation of $P=\langle R,\textbf{I}\rangle$. We distinguish inductive cases by different rules (R1-R4)
that are possibly applied to derive $R(a,b)$.

\emph{Basic case}. If $R(a,b)\in\textbf{I}$, $R(a,b)\in\delta_{R}$. In this case,
regardless of the case that $R$ is a TI or an NTI role, all of (1), (2) and (3) hold.

\emph{Inductive case}. We first study the case where $R$ is an NTI role.

\begin{enumerate}[leftmargin=12ex,label=Case~1.\arabic*, ref=Case~1.\arabic*]
\item $R(a,b)$ is derived by applying the rule (R1), w.l.o.g., $R'(a,b)\rightarrow R(a,b)$,
    which also means that $R'\sqsubseteq R\in\mathcal{R}$. Since $R$ is an NTI role, $R'$ has to
    be an NTI role according to Definition~\ref{def:deltaRdhlplus}. By the induction hypothesis,
    $R'(a,b)\in\delta_{R'}$ holds. (Case~1.1.1) If $R'(a,b)\in\mathcal{A}$, $R(a,b)\in\delta_{R}$
    holds according to Definition~\ref{def:deltaRdhlplus}; (Case~1.1.2) If there exists some
    role $R''$ such that $R''(a,b)\in\mathcal{A}$ and $R''\sqsubseteq R'$, we also have that
    $R(a,b)\in\delta_{R}$ because $R''\sqsubseteq_* R$ holds; (Case~1.1.3) If there exists some
    role $R''$ such that $R''(b,a)\in\mathcal{A}$ and $R''\sqsubseteq R'^-$, this case is
    similar to (Case~1.1.2); (Case~1.1.4) There exists $R_0R_2,...,R_n\in\mathcal{L}(R')$,
    where $R_i(x_i,x_{i+1})\in\mathcal{A}$ and $x_0=a, x_{n+1}=b$ for $0\leq i\leq n$.
    According to Definition~\ref{def:language}, $R_0R_2,...,R_n\in\mathcal{L}(R)$ holds since
    $R'\sqsubseteq R\in\mathcal{R}$.\label{dhlplus:NTIcaseR1}

\item $R(a,b)$ is derived by applying the rule (R2), w.l.o.g., $R'(a,b)\rightarrow R(b,a)$,
    which also means that $R'\sqsubseteq R^-\in\mathcal{R}$. Since $R$ is an NTI role, $R'$ has to
    be an NTI role according to Definition~\ref{def:deltaRdhlplus}. By the induction hypothesis,
    $R'(b,a)\in\delta_{R'}$. (Case~1.2.1) If $R'(b,a)\in\mathcal{A}$, $R(a,b)\in\delta_{R}$
    holds according to Definition~\ref{def:deltaRdhlplus}; (Case~1.2.2) If there exists some
    role $R''$ such that $R''(b,a)\in\mathcal{A}$ and $R''\sqsubseteq R'$, we also have that
    $R(a,b)\in\delta_{R}$ because $R''\sqsubseteq_* R^-$ holds; (Case~1.2.3) If there exists some
    role $R''$ such that $R''(a,b)\in\mathcal{A}$ and $R''\sqsubseteq R'^-$, this case is
    similar to (Case~1.2.2); (Case~1.2.4) there exists $R_0R_2,...,R_n\in\mathcal{L}(R')$,
    where $R_i(x_i,x_{i+1})\in\mathcal{A}$ and $x_0=b, x_{n+1}=a$ for $0\leq i\leq n$.
    According to Definition~\ref{def:language}, $R_n^-,...,R_1^-\in\mathcal{L}(R)$ holds since
    $R'\sqsubseteq R^-\in\mathcal{R}$.\label{dhlplus:NTIcaseR2}

\item $R(a,b)$ is derived by applying the rule (R4), w.l.o.g.,
    $R_1(a,c_2),R_2(c_2,b)\rightarrow R(a,b)$,
    which also means that $R_1\circ R_2\sqsubseteq R\in\mathcal{R}$.
    Since $R$ is an NTI role, both of $R_1$ and $R_2$ are NTI roles. By the induction hypothesis,
    for each $R_i(c_i,c_{i+1})(1\leq i\leq 2)$ where $c_1=a, c_3=b$, $R_i(c_i,c_{i+1})\in\delta_{R_i}$
    hold. We have a role sequence $L_i\in\mathcal{L}(R_i)$ that may be constructed in following different cases:

    \begin{enumerate}[leftmargin=14ex,label=Case~1.3.\arabic*]
    \item If there exists a role $R'_i$ such that $R'_i(c_i,c_{i+1})\in\mathcal{A}$
        and $R'_i\sqsubseteq_* R_i$, then we have that $L_i=R'_i$.

    \item If there exists a role $R'_i$ such that $R'_i(c_{i+1},c_{i})\in\mathcal{A}$
        and $R'_i\sqsubseteq_* R_i^-$, then we have that $L_i=R^{'-}_i$.

    \item If there exists a role sequence $R_0R_2,...,R_m\in\mathcal{L}(R')$, where $R_j(x_j,x_{j+1})\in\mathcal{A}$ and $x_0=c_i, x_{n+1}=c_{i+1}$ for $0\leq i\leq m$
        and $R'\sqsubseteq_* R_i$, then $L_i=R_0R_2,...,R_m$ holds.
    \end{enumerate}

    Based on $L_i(1\leq i\leq 2)$, we have that $L_1L_2\in\mathcal{L}(R)$. Further $R(a,b)\in\delta_R$
    holds.\label{dhlplus:NTIcaseR4}

\item $R(a,b)$ is derived by applying the rule (R3). This case is impossible,
    since $R$ is an NTI but not transitive role.
\end{enumerate}

We next study the case where $R$ is a transitive role.

\begin{enumerate}[leftmargin=12ex,label=Case~2.\arabic*, ref=Case~2.\arabic*]
\item $R(a,b)$ is derived by applying the rule (R1), w.l.o.g., $R'(a,b)\rightarrow R(a,b)$,
    which also means that $R'\sqsubseteq R\in\mathcal{R}$.

    \begin{enumerate}[leftmargin=8ex,label=Case~2.1.\arabic*]
    \item $R'$ is an NTI role. By the induction hypothesis, $R'(a,b)\in\delta_{R'}$.
        Similar to \ref{dhlplus:NTIcaseR1}, $R(a,b)\in\delta_{R}$ holds and
        obviously, $R\in\mathcal{L}(R)$ holds.

    \item $R'$ is a TI but not a transitive role.
        By the induction hypothesis, there exists $R_1,...,R_n\in\mathcal{L}(R')$
        such that $R_i(x_i,x_{i+1})\in\delta_{R_i}$ or $R_i(x_i,x_{i+1})\in\delta^*_{R_i}$, where $1\leq i\leq n$
        and $x_0=a, x_{n+1}=b$. Since $R'\sqsubseteq R$ holds, $R_1,...,R_n\in\mathcal{L}(R)$ also holds.

    \item $R'$ is a transitive role.
        By the induction hypothesis, there exists $R'(a,b)\in\delta^*_{R'}$.
        Further, let $R'(a,c_1),R'(c_1,c_2),...,R'(c_n,b)\in\delta_{R'}$.
        Since $R'\sqsubseteq R\in\mathcal{R}$ holds, we also have that
        $R(a,c_1),R(c_1,c_2)$ $,...,R(c_n,b)\in\delta_{R}$. Obviously $R(a,b)\in\delta^*_{R}$ holds.
    \end{enumerate}

\item $R(a,b)$ is derived by applying the rule (R2), w.l.o.g., $R'(b,a)\rightarrow R(a,b)$,
    which also means $R'\sqsubseteq R^-\in\mathcal{R}$.

    \begin{enumerate}[leftmargin=8ex,label=Case~2.2.\arabic*]
    \item $R'$ is an NTI role. By the induction hypothesis, $R'(b,a)\in\delta_{R'}$.
        Similar to \ref{dhlplus:NTIcaseR2}, $R(a,b)\in\delta_{R}$ holds and
        obviously, $R\in\mathcal{L}(R)$ holds.

    \item $R'$ is a TI but not transitive role. By the induction hypothesis, there exists $R_1,...,R_n\in\mathcal{L}(R')$
        such that $R_i(x_i,x_{i+1})\in\delta_{R_i}$ or $R_i(x_i,x_{i+1})\in\delta^*_{R_i}$, where $1\leq i\leq n$
        and $x_0=b, x_{n+1}=a$. Since $R'\sqsubseteq R^-$ holds, $R_n^-,...,R_1^-\in\mathcal{L}(R)$ also holds.

     \item $R'$ is a transitive role.
        By the induction hypothesis, there exists $R'(b,a)\in\delta^*_{R'}$.
        Further, let $R'(b,c_1),R'(c_1,c_2),...,R'(c_n,a)\in\delta_{R'}$.
        Since $R'\sqsubseteq R^-\in\mathcal{R}$, we also have that
        $R(a,c_1),R(c_1,c_2)$ $,...,R(c_n,b)\in\delta_{R}$. Obviously $R(a,b)\in\delta^*_{R}$ holds.
    \end{enumerate}

\item $R(a,b)$ is derived by applying the rule (R4), w.l.o.g.,
    $R_1(a,c_2),R_2(c_2,b)\rightarrow R(a,b)$,
    which also means that $R_1\circ R_2\sqsubseteq R\in\mathcal{R}$.
    Further, $R$ is a transitive role. According to the simple-role restriction,
    both of $R_1$ and $R_2$ are simple roles.
    This also
    means that $R_1, R_2$ are NTI roles. Further, $R_1(a,c_2)\in\delta_{R_1}$ and $R_2(c_2,b)\in\delta_{R_2}$.
    This case is similar to \ref{dhlplus:NTIcaseR4}.

\item $R(a,b)$ is derived by applying the rule (R3), w.l.o.g., $R(a,c),R(c,b)\rightarrow R(a,b)$,
    which also means that $R\circ R\sqsubseteq R\in\mathcal{R}$.
    Since $R$ is a TI role, by the induction hypothesis,
    $R(a,c),R(c,b)\in\delta^*_{R}$ holds. Obviously, $R(a,b)\in\delta^*_{R}$ holds.
\end{enumerate}

We finally study the case where $R$ is a TI but not a transitive role.

\begin{enumerate}[leftmargin=12ex,label=Case~3.\arabic*, ref=Case~3.\arabic*]
\item $R(a,b)$ is derived by applying the rule (R1), w.l.o.g., $R'(a,b)\rightarrow R(a,b)$,
    which also means that $R'\sqsubseteq R\in\mathcal{R}$.\label{dhlplus:TIcaseR1}

    \begin{enumerate}[leftmargin=8ex,label=Case~3.1.\arabic*]
    \item $R'$ is an NTI role. By the induction hypothesis, $R'(a,b)\in\delta_{R'}$ holds.
        Similar to \ref{dhlplus:NTIcaseR1}, $R(a,b)\in\delta_{R}$ holds and
        obviously, $R\in\mathcal{L}(R)$ holds.

    \item $R'$ is a TI but not a transitive role.
        By the induction hypothesis, there exists $R_1,...,R_n\in\mathcal{L}(R')$
        such that $R_i(x_i,x_{i+1})\in\delta_{R_i}$ or $R_i(x_i,x_{i+1})\in\delta^*_{R_i}$, where $1\leq i\leq n$
        and $x_0=a, x_{n+1}=b$. Since $R'\sqsubseteq R$, $R_1,...,R_n\in\mathcal{L}(R)$ also holds.

    \item $R'$ is a transitive role.
        By the induction hypothesis, there exists $R'(a,b)\in\delta^*_{R'}$.
        According to Definition~\ref{def:language}, $R'\in\mathcal{L}(R)$ holds.
        Then, $R'$ is actually the role sequence that satisfies the condition
        in this lemma.
    \end{enumerate}

\item $R(a,b)$ is derived by applying the rule (R2), w.l.o.g., $R'(b,a)\rightarrow R(a,b)$,
    which also means that $R'\sqsubseteq R^-\in\mathcal{R}$.

    \begin{enumerate}[leftmargin=8ex,label=Case~3.2.\arabic*]
    \item $R'$ is an NTI role. By the induction hypothesis, $R'(b,a)\in\delta_{R'}$.
        Similar to \ref{dhlplus:NTIcaseR2}, $R(a,b)\in\delta_{R}$ holds and
        obviously, $R\in\mathcal{L}(R)$ holds.

    \item $R'$ is a TI but not a transitive role. By the induction hypothesis, there exists $R_1,...,R_n\in\mathcal{L}(R')$
        such that $R_i(x_i,x_{i+1})\in\delta_{R_i}$ or $R_i(x_i,x_{i+1})\in\delta^*_{R_i}$, where $1\leq i\leq n$
        and $x_0=b, x_{n+1}=a$. Since $R'\sqsubseteq R^-$, $R_n^-,...,R_1^-\in\mathcal{L}(R)$ also holds.

     \item $R'$ is a transitive role.
        By the induction hypothesis, there exists $R'(b,a)\in\delta^*_{R'}$.
        Similarly to \ref{dhlplus:TIcaseR1} (Case~1.1.3), we have that
        $R'^-$ is actually the role sequence that satisfies the condition
        in this lemma.
    \end{enumerate}

\item $R(a,b)$ is derived by applying the rule (R4), w.l.o.g.,
    $R_1(a,c_2),R_1(c_2,b)$ $\rightarrow R(a,b)$,
    which also means that $R_1\circ R_2\sqsubseteq R\in\mathcal{R}$ holds.
    Further, $R$ is a TI but not transitive role. According to the simple-role restriction,
    at least one of $R_1$ and $R_2$ is a simple role.
    W.l.o.g., let $R_2$ be a simple role, which also
    means that $R_2$ is an NTI role. Further, $R_2(c_2,b)\in\delta_{R_2}$ holds.
    For $R_1$, by the induction hypothesis, $R_1(a,c_2)\in\delta^*_{R_1}$ holds.
    On the other hand, a role sequence $L_{R_2}$ exists in $\mathcal{L}(R_2)$ such that $c_2$ and $b$
    are the starting and ending individual respectively.
    Further, we have that $R_1L_{R_2}\in\mathcal{L}(R)$ which satisfies the second condition
    in Lemma~\ref{lemma:tiplus}.\label{dhlplus:TIcaseR4}

\item $R(a,b)$ is derived by applying the rule (R3), w.l.o.g., $R(a,c),R(c,b)\rightarrow R(a,b)$,
    which also means that $R\circ R\sqsubseteq R\in\mathcal{R}$.
    Since $R$ is not a transitive role, this case is impossible.\hfill$\Box$
\end{enumerate}

\section{Proof of Theorem~\ref{theorem:aprc}}

\textbf{Theorem~\ref{theorem:aprc}}
\emph{For any DHL$(\circ)$ ontology $\mathcal{O}$, Algorithm~$\mathsf{A}_{prc}$ halts and outputs
a materialization graph $\mathcal{O}$.}\\

\noindent\emph{Proof}: similar to the proof of Lemma~\ref{lemma:a3},
this theorem is proved in two stages:
(1) the graph $\mathcal{G}$ returned by $\mathsf{A}_{prc}$ is a materialization graph;
(2) $\mathcal{G}$ is a complete materialization graph.
One can show that (1) holds by performing the same induction on the iterations of \ref{alg3:updateG} of $\mathsf{A}_{opt}$
(see the proof of Lemma~\ref{lemma:a3}). This is because that
such an induction is performed regardless of how the relation $S_{\textit{\!\tiny rch}}$ is computed.

To prove that (2) holds, we conduct the induction used in the proof for Lemma~\ref{lemma:a1}.
We aim to show that all the ground atoms in $T_R^{i+1}(\textbf{I})$ have to be added to $\mathcal{G}$,
with the induction hypothesis that the ground atoms in $T_R^{i}(\textbf{I})$ are in $\mathcal{G}$.
Suppose the ground atoms in $T_R^{i}(\textbf{I})$ have been added to $\mathcal{G}$ by
performing $\mathsf{A}_{prc}$.
For each atom $\alpha$ in $T_R^{i+1}(\textbf{I})$, there has to be a relation $(\beta,\alpha)$
in $S_{\textit{\!\tiny rch}}$ obtained in some iteration; further
$\beta$ is already in $\mathcal{G}$ according to the inductive hypothesis. Thus,
$\alpha$ has to be added to $\mathcal{G}$ by applying \ref{alg4:updateG}
of $\mathsf{A}_{prc}$.\hfill$\Box$


\section{Proof of Theorem~\ref{theorem:aprcpt}}

\textbf{Theorem~\ref{theorem:aprcpt}}
\emph{For any DHL$(\circ)$ ontology $\mathcal{O}$ that follows the simple-concept and the simple-role
restrictions,
there exists a poly-logarithmically bounded function $\psi$,
such that $\mathsf{A}_{prc}^{\psi}$ outputs
a materialization graph of $\mathcal{O}$.}\\

\noindent\emph{Proof}: similar to what we do when proving
Theorem~\ref{theorem:dhl} and Theorem~\ref{theorem:dhlplus},
we separate the materialization of a DHL$(\circ)$ ontology
into two stages: the role materialization (all the rules
of the forms (R1-R4) are exhaustively applied) and the concept materialization
(the rules of the forms (T1-T3)
are then applied). The difference is that, we further separate the role materialization
into successively two stages: \emph{the simple role materialization} (Stage~SRM) and
\emph{the non-simple role materialization} (Stage~NSRM).
Similarly, the concept materialization is also separated into successively two stages:
\emph{the simple concept materialization} (Stage~SCM) and
\emph{the non-simple concept materialization} (Stage~NSCM).
Specifically, in Stage~SRM, all atoms of the form $R(a,b)$ are derived
where $R$ is a simple role, while all atoms of the form $R(a,b)$ for the non-simple
role $R$ are derived in Stage~NSRM. Stage~SCM and Stage~NSCM can be explained similarly.

We aim to prove that, there always exists a poly-logarithmically bounded function $\psi$
such that $\mathsf{A}_{prc}^{\psi}$ can handle each of the four stages, Stage~SRM,
Stage~NSRM, Stage~SCM and Stage~NSCM. Based on the previous result, we have that
there exists a poly-logarithmically bounded function $\psi'$
such that $\mathsf{A}_{prc}^{\psi'}$ handles the whole materialization.

We first consider Stage~SCM.
Stage~SCM (the simple concept materialization) is conducted on axioms of two forms $A\sqsubseteq B$
and $\exists R.A\sqsubseteq B$, which correspond to datalog rules of the forms $A(x)\rightarrow B(x)$
and $R(x,y),A(y)\rightarrow B(x)$ respectively. Since role assertions are supposed to be fixed
during in Stage~SCM, role $R$ in each rule of the form $R(x,y),A(y)\rightarrow B(x)$
can be viewed as an EDB predicate. This further means that,
for each atom of the form $A(a)$ where $A$ is a simple concept,
$A(a)$ has an SWD path in the first iteration of $\mathsf{A}_{prc}$.
Recall Algorithm~$\mathsf{Prc}$. We use an induction to show that
such an SWD path of $A(a)$ can be determined by $S_{\textit{\!\tiny rch}}$.
The atom $A(a)$ can be derived through either of the following rule instantiations:

\begin{enumerate}[leftmargin=12ex]
\item[Case~1] $B(a)\rightarrow A(a)$, where $\mathcal{O}\models B\sqsubseteq_* A$. We have
    that \texttt{rch}$(B(a),A(a))$ is in $S_{\textit{\!\tiny rch}}$ according to Algorithm~$\mathsf{Prc}$.
    By the induction hypothesis, the existence of the SWD path for $B(a)$ can be determined.
    Thus, the existence of the SWD path of $A(a)$ can also be determined by $S_{\textit{\!\tiny rch}}$.

\item[Case~2] $R(a,b),B(b)\rightarrow A(a)$ where $\exists R.B\sqsubseteq A\in\mathcal{T}$.
    According to Algorithm~$\mathsf{Prc}$ and the fact that $R(a,b)$ has to be derived,
    we have that \texttt{rch}$(B(b),$ $A(a))$ is in $S_{\textit{\!\tiny rch}}$.
    By the induction hypothesis, the existence of the SWD path for $B(b)$ can be determined.
    Thus, the existence of the SWD path of $A(a)$ can also be determined.
\end{enumerate}

We next discuss the non-simple concept materialization (Stage~NSCM), which is conducted on axioms of forms (T1-T3).
According to the simple-concept restriction, for each axiom of the form $A_1\sqcap A_2\sqsubseteq B$,
one of $A_1$ and $A_2$ should be simple concept and can be viewed as an EDB predicate, since Stage~SCM is completed.
In Stage~NSCM, for each atom of the form $A(a)$ where $A$ is not a simple concept,
$A(a)$ has an SWD path in the first iteration of $\mathsf{A}_{prc}$ due to the simple concept restriction.
We also use an induction to show that
such an SWD path of $A(a)$ can be determined by distinguishing different rule instantiations.
The case of the rule instantiation of the form $B(a)\rightarrow A(a)$ (resp., $R(a,b),B(b)\rightarrow A(a)$)
is referred to Case~1 (resp., Case~2). We only consider the case of the rule instantiation of the form $A_1(a),A_2(a)\rightarrow A(a)$
here.

\begin{enumerate}[leftmargin=12ex]
\item[Case~3] $A_1(a),A_2(a)\rightarrow A(a)$ where $A_1\sqcap A_2\sqsubseteq A\in\mathcal{T}$.
    If $A_1(a)$ is derived where $A_1$ is a simple concept,
    we have that \texttt{rch}$(A_2(a),A(a))$ is in $S_{\textit{\!\tiny rch}}$
    according to Algorithm~$\mathsf{Prc}$.
    By the induction hypothesis, the existence of the SWD path for $A_2(a)$ can be determined.
    Thus, the existence of the SWD path of $A(a)$ can also be determined.
    It is similar for the case where $A_2(a)$ is derived and $A_2$ is a simple concept.
\end{enumerate}

The simple-role materialization (Stage~SRM) can be simply conducted by checking the transitive closure
of role inclusions, i.e., the axioms of the forms (R1-R2).
It is easy to check that the simple-role materialization can be finished after the
first iteration of $\mathsf{A}_{prc}$.

We now discuss Stage~NSRM. For each atom of the form $R(a,b)$ where $R$ is not a simple role
and not a transitive role,
$R(a,b)$ has an SWD path in the first iteration of $\mathsf{A}_{prc}$ due to the simple role restriction.
We use an induction to show that
such an SWD path of $R(a,b)$ can be determined by distinguishing different rule instantiations
as follows.

\begin{enumerate}[leftmargin=12ex]
\item[Case~4] $S(a,b)\rightarrow R(a,b)$ where $\mathcal{O}\models S\sqsubseteq_* R$.
    We have that \texttt{rch}$(S(a,b),R(a,b))$ is in $S_{\textit{\!\tiny rch}}$
    according to Algorithm~$\mathsf{Prc}$.
    By the induction hypothesis, the existence of the SWD path for $S(a,b)$ can be determined.
    Thus, the existence of the SWD path of $R(a,b)$ can also be determined.

\item[Case~5] $S(b,a)\rightarrow R(a,b)$ where $\mathcal{O}\models S\sqsubseteq_* R^-$.
    We have that \texttt{rch}$(S(b,a),R(a,b))$ is in $S_{\textit{\!\tiny rch}}$
    according to Algorithm~$\mathsf{Prc}$.
    By the induction hypothesis, the existence of the SWD path for $S(b,a)$ can be determined.
    Thus, the existence of the SWD path of $R(a,b)$ can also be determined.

\item[Case~6] $R_1(a,c),R_2(c,b)\rightarrow R(a,b)$ where $R_1\circ R_2\sqsubseteq R\mathcal{R}$.
    If $R_1(a,c)$ is derived where $R_1$ is a simple role,
    we have that \texttt{rch}$(R_2(c,b),R(a,b))$ is in $S_{\textit{\!\tiny rch}}$
    according to Algorithm~$\mathsf{Prc}$.
    By the induction hypothesis, the existence of the SWD path for $R_2(c,b)$ can be determined.
    Thus, the existence of the SWD path of $R(a,b)$ can also be determined.
    It is similar for the case where $R_2(c,b)$ is derived and $R_2$ is a simple role.
\end{enumerate}

We finally consider the derivation of the atoms of the form $R(a,b)$, where $R$
is a transitive role.
Recall Lemma~\ref{lemma:tiplus}. We have that, $R(a,b)\in\delta_R$ or
$R(a,b)\in\delta^*_R$ holds.
According to the simple role restriction, for each sub-role $S$ of $R$, $S$ is a simple role,
or there exists such axioms
of the form $S_1\circ S_2\sqsubseteq S$ where $S_1$ and $S_2$ are simple roles.
One can check the set $\delta_R$ can computed by $\mathsf{A}_{prc}$ in
at most two iterations (see Case~4, Case~5 and Case~6).
We should prove that the transitive closure $\delta^*_R$
can also be computed by $\mathsf{A}_{prc}^\psi$ for some
poly-logarithmically bounded function $\psi$.

\begin{enumerate}[leftmargin=12ex]
\item[Case~7] $R(a,c),R(c,b)\rightarrow R(a,b)$ where $R$ is a transitive role.
    One can construct a binary tree $t$ where the root node is $R(a,b)$,
    the leaves are in $\delta_R$ and the hight of $t$ is up-bounded by
    $\log(|\delta_R|)$. It can be checked that, $\mathsf{A}_{prc}$ can generate
    all nodes in each level of $t$ in one iteration from the bottom. Thus,
    $R(a,b)$ can be derived in at most $\log(|\delta_R|)$ iterations.\hfill$\Box$
\end{enumerate}




%%% Local Variables:
%%% mode: latex
%%% TeX-master: "parallel-tractability-J"
%%% End:
